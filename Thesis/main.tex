% Szkielet dla pracy licencjackiej pisanej w języku polskim.

\documentclass[polish,bachelor,a4paper,oneside]{ppfcmthesis}

\usepackage[utf8]{inputenc}
\usepackage[OT4]{fontenc}
\usepackage{amsmath}
\usepackage{float}
\usepackage{caption}
\usepackage{listings}
\usepackage{color}
\usepackage{subfig}
\definecolor{mauve}{rgb}{0.58,0,0.82}
\definecolor{dkgreen}{rgb}{0,0.3,0}


%--------------------------------------
% Strona tytułowa
%--------------------------------------

% Autorzy pracy, jeśli jest ich więcej niż jeden
% wstaw między nimi separator \and
\author{%
   Arkadiusz Chmura \album{136690} \and 
   Bartosz Ciesielski \album{136694}\and
   Iwo Naglik \album{136774}\and
   Bartosz Przybył \album{136785}}
\authortitle{}                                % Do not change.

\title{Algorytmy uczenia maszynowego w predykcji wyników meczów piłkarskich}

% Your supervisor comes here.
\ppsupervisor{dr hab. inż. Jerzy Stefanowski, prof. nadzw. PP} 

% Year of final submission (not graduation!)
\ppyear{2021}                                 


\begin{document}

\lstset{
  language=Python,
  aboveskip=4mm,
  belowskip=4mm,
  showstringspaces=false,
  columns=flexible,
  basicstyle={\small\ttfamily},
  keywordstyle=\color{dkgreen},
  commentstyle=\color{blue},
  stringstyle=\color{mauve},
  breaklines=true,
  breakatwhitespace=true,
  tabsize=3
}

% Front matter starts here
\frontmatter\pagestyle{empty}%
\maketitle\cleardoublepage%


%--------------------------------------
% Podziękowania
\newpage\null\thispagestyle{empty}\newpage
\begin{center}
    \huge Podziękowania
\end{center}
Składamy serdeczne podziękowania naszemu Promotorowi, dr hab. inż. Jerzemu Stefanowskiemu prof. nadzw. PP za cenne uwagi przekazywane od początku współpracy. W szczególności za wszechstronną pomoc, która była dużym wsparciem podczas pisania pracy, a także za cały poświęcony czas podczas tworzenia projektu.

%--------------------------------------

%--------------------------------------
% Streszczenie
\newpage\null\thispagestyle{empty}\newpage
\newpage
\begin{center}
    \huge Streszczenie
\end{center}
Analiza danych sportowych, w szczególności przewidywanie wyników w rozgrywkach sportowych, jest przedmiotem zainteresowania praktyków i badaczy od wielu lat. Pomimo zaproponowania już wielu narzędzi wspomagających predykcję wyniku meczu, uzyskiwane rezultaty wciąż pozostawiają przestrzeń na poprawę i doskonalenie. W niniejszej pracy podjęto próbę stworzenia systemu, który pozwoliłby wskazać wynik nadchodzącego meczu piłkarskiego pomiędzy dwoma drużynami z pierwszej ligi angielskiej na podstawie starannie zagregowanych danych pochodzących z wielu źródeł. Architektura systemu zawiera bazę danych, web API udostępniające dane, moduł związany zarówno ze wstępnym przetwarzaniem, jak i selekcją cech oraz przygotowujący dane w odpowiednim formacie. Utworzono również środowisko \textit{Jupyter Notebook}, w którym nauczone zostają odpowiednie modele klasyfikacyjne, na których później zostaje przeprowadzona predykcja końcowa. Przeprowadzono obszerne badania eksperymentalne wybranych algorytmów uczenia klasyfikatorów oraz znaczenia atrybutów dla klasyfikacji. Stworzono także interaktywny notebook, dla potencjalnych analityków sportowych, umożliwiający wybór drużyn rozgrywających dane spotkanie oraz algorytmu, według którego ma zostać przeprowadzona ostateczna predykcja wyniku.\\\\

 

\begin{center}
    \huge Abstract
\end{center}
The analysis of sports data, in particular the prediction of results in sports competitions, has been of interest to practitioners and researchers for many years. Despite the development of many tools supporting the prediction of the result of a sport match, the obtained outcomes still leave room for improvement. In this thesis, an attempt was made to create a system which would predict the result of an upcoming match between two teams from the English Premier League on the basis of carefully aggregated data from many sources. Archtecture of the system includes a database, web API that provides data, python module responsible for preprocessing, feature selection and preparation of data in a proper format and a \textit{Jupyter Notebook} environment in which trained classification models are used to conduct final prediction. Extensive experimental studies of selected classifier learning algorithms and the significance of attributes for classification were conducted. An interactive notebook was also created for potential sports analysts, which allow them to choose teams playing a given match and an algorithm by which the final prediction of the result is to be carried out.

%--------------------------------------

%--------------------------------------
% Spis treści
%--------------------------------------
\newpage\null\thispagestyle{empty}\newpage
\newpage
\pagenumbering{Roman}\pagestyle{ppfcmthesis}%
\tableofcontents* 
\cleardoublepage % Zaczynamy od nieparzystej strony
%--------------------------------------
% Rozdziały
%--------------------------------------

%Najwygodniej jeśli każdy rozdział znajduje się w oddzielnym pliku
\mainmatter%

\chapter{Wstęp: Sztuczna inteligencja w analizie danych o rozgrywkach sportowych}

\noindent Praca podejmuje tematykę eksploracji danych o rozgrywkach sportowych i wykorzystania algorytmów uczenia maszynowego do predykcji możliwych wyników rozgrywek. Tematyka tak rozumianej analizy danych sportowych z jednej strony była i jest coraz intensywniej rozważana przez trenerów, a z drugiej strony stała się przedmiotem zainteresowania badaczy w ostatnich kilkunastu latach - co widać po rosnącej liczbie publikacji dotyczących zróżnicowanych dyscyplin sportowych (m.in piłki nożnej~\cite{Euro2016-1}, koszykówki~\cite{basketball}, hokeja na lodzie~\cite{ice-hockey}, kolarstwa~\cite{cyclists}, czy pływania~\cite{swimming}). 

Wśród badanych zastosowań powyższych metod można zwrócić uwagę na popularność dyscypliny piłki nożnej \cite{Euro2016-1} \cite{Euro2016-2} \cite{Euro2016-3} \cite{soccer_players_skill} \cite{ml_soccer_analytics}. Także w niniejszej pracy zainteresowano się możliwością poszukania i pozyskania danych nt. rozgrywek meczów piłkarskich z ligi angielskiej - najczęściej oglądanej w tej dyscyplinie sportu~\cite{ESPN} - oraz zbadania przydatności różnych metod uczenia maszynowego w przewidywaniu rezultatu przyszłych meczy na podstawie historycznych danych z ostatnich sezonów. 

%\noindent Praca podejmuje tematykę eksploracji danych pochodzących z różnych źródeł i występujących w różnych strukturach oraz zastosowań metod uczenia maszynowego w przewidywaniu rezultatu meczów piłkarskich z ligi angielskiej - najczęściej oglądanej w tej dyscyplinie sportu~\cite{ESPN}.

Zauważmy, ze popularność tej ligi i samej dyscypliny powoduje powstawanie ogromnej ilości danych możliwych do analiz, co czyni ją atrakcyjną dla badaczy. W szczególności przewidywanie wyniku przejawia się dużym zainteresowaniem ze względu na wspomnianą dostępność danych oraz wzrastającą liczbę zakładów bukmacherskich, z którymi można grać o prawdziwe pieniądze. To z kolei przyciąga nie tylko naukowców chcących tworzyć narzędzia i metody pozwalające na trafne określenie zwycięzcy meczu, ale też fanów tej dyscypliny, którzy stawiają swoje pieniądze z nadzieją na zysk dzięki podpowiadającej im wynik intuicji. 

Poszukiwanie nowych rozwiązań i metod nie należy wyłącznie do działań  twórców zakładów bukmacherskich, którzy, aby stale generować zyski, muszą ulepszać swoje metody predykcji, ale też przede wszystkim dla działaczy klubu (a w szczególności analityków) chcących przewidywać kto jest faworytem danego spotkania, aby w razie potrzeby móc odpowiednio dostosować taktykę na dany pojedynek, czy dobrać zespół.

Ponadto studiując literaturę można zauważyć, że turnieje międzynarodowe w piłce nożnej stały się dobrym poligonem treningowym  dla naukowców chcących przetestować swoje rozwiązania. Ostatnie rozgrywki Mistrzostw Europy w 2016 wraz z udostępnieniem publicznie otwartych danych doprowadziły do powstania wielu artykułów naukowych  prezentujących nowe podejścia w tej dziedzinie, co pokazuje jak duże zainteresowanie panuje w tym temacie~\cite{Euro2016-1} \cite{Euro2016-2} \cite{Euro2016-3}.

Należy jednak zauważyć, ze przewidzenie zwycięzcy meczu jest bardzo trudnym zadaniem, gdyż wpływa na niego wiele czynników i większość z nich jest związana z czynnikiem ludzkim. Ponadto wyzwaniem jest zlokalizowanie właściwych repozytoriów danych, często bardzo zróżnicowanych (np. oprócz samych zapisów wyników kolejnych sezonów trzeba poszukiwać dodatkowych repozytoriów nt. samych graczy, oceny tzw. potencjału drużyny, a niektórzy autorzy sugerują wykorzystywanie różnych wskaźników stosowanych przez firmy bukmacherskie). Powyższe źródła danych są na ogół reprezentowane w różnych formatach, co wymaga wysiłku do ich poprawnej integracji, a później zastosowania odpowiednich metod wstępnego przetwarzanie w celu wydobycia najważniejszych danych zarówno o meczu, jak i graczach biorących w nim udział, aby przewidzieć kto ostatecznie zwycięży w danym spotkaniu. Same algorytmy uczenia maszynowego były rozwijane dla innych zastosowań, a wyniki zamieszczone w literaturze nie są często o wysokiej trafności. Ponadto zauważamy, że ciągle wymyślane są nowe podejścia oraz udoskonalane obecne algorytmy. 

Kierując się powyższymi motywacjami w niniejszej rozprawie podjęto próbę utworzenia całościowego system, który będzie zarówno integrował różnorodne dane o rozgrywkach piłkarskich ligi angielskiej, a przede wszystkim stosował wybrane algorytmy uczenia maszynowego do przewidywania, która drużyna zostanie zwycięzcą danego pojedynku. 


\chapter{Cel i zakres pracy}
\section{Wykorzystane narzędzie i technologie}
\section{Podział pracy}

\chapter{Przegląd badań na temat eksploracji danych sportowych oraz pojęcia podstawowe}
Temat eksploracji danych sportowych i wykorzystania metod z dziedziny uczenia maszynowego w sporcie rozwinął się znacząco w ostatnich latach. Świadczą o tym chociażby organizowane corocznie konferencje poświęcone szeroko rozumianej analizie sportu, takie jak MLSA (\textit{Machine Learning and Data Mining for Sports Analytics}) przeprowadzane co roku w innym kraju, czy \textit{MIT Sloan Sports Analytics Conference}, która ma miejsce w Stanach Zjednoczonych. Konferencje te zrzeszają nie tylko naukowców i badaczy, ale też trenerów, działaczy klubów, a nawet samych zawodników.

W przeszłości prace naukowe dotyczące piłki nożnej pojawiały się rzadziej od prac traktujących o innych dyscyplinach sportowych~\cite{ml_soccer_analytics}. Wynika to z zupełnie innej natury tego sportu. Jak wspomniano w pracy~\cite{game_classification}, piłka nożna należy do sportu z kategorii terytorialnych (\english{territory games}), które zakładają kontrolę pewnego obiektu, trzymanie go z dala od przeciwnika i stopniowego przesuwania się do pozycji, która pozwoli zdobyć punkt/bramkę. Zawodnicy zarówno defensywni, jak i ofensywni zajmują ten sam obszar boiska próbując jednocześnie uniemożliwić zdobycie punktów przez przeciwnika. Taki styl gry różni się nieco od np. baseballu, który należy do kategorii określonej jako \textit{Striking/Fielding} i charakteryzuje się zdobywaniem punktu poprzez wykonanie rzutu i biegnięcie do wyznaczonego miejsca lub uniemożliwieniem zdobycia punktu przez przeciwnika poprzez przechwycenie tego rzutu. Mając na uwadze różnice w stylach tych dyscyplin, można zauważyć, iż łatwiej jest rozbić pojedynczy mecz w baseballu na dyskretne wydarzenia do analizy, a dynamiczna i skomplikowana struktura piłki nożnej czyni taką analizę znacznie trudniejszą.

Potrzeba dostępu do szczegółowych i wyrafinowanych analiz meczów piłki nożnej spowodowała powstanie wielu komercyjnych firm takich jak \textit{Wyscout} czy \textit{StatsBomb}, które dostarczają wyszukanych analiz wideo dzielących każde mecze na czynniki pierwsze przy pomocy technik rozpoznawania obrazów (\english{computer vision}) i uwzględniają niewidoczne dla ludzkiego oka szczegóły- dokładną pozycję bramkarza w momencie utraty bramki, siłę strzału, pozycję zawodnika, od którego rozpoczęła się akcja bramkowa itd.

Mimo, iż techniki uczenia maszynowego w sporcie (a w szczególności w piłce nożnej) są najczęściej wykorzystywane w celu przewidywania zwycięzcy spotkania (chociażby prace z ostatniego dużego turnieju europejskiego - Euro 2016~\cite{Euro2016-1} \cite{Euro2016-2} \cite{Euro2016-3}), wiele dzieł dotyczy także innych zagadnień. Podejmowane są przykładowo tematy ewaluacji umiejętności danego zawodnika (oraz jego szans na dalszy rozwój) i całej drużyny~\cite{ml_soccer_analytics} \cite{soccer_players_skill}, które mogą potencjalnie wspomóc łowców talentów szukających nowych zawodników do drużyn, czy też automatycznego wykrywania ludzkiej aktywności ruchowej~\cite{activity_recignition}, które może umożliwić na przykład znalezienie optymalnych pozycji zawodników do wykonywania strzału, podania czy sprintu.

Zrealizowanie celu tej pracy, który zakłada zbudowanie systemu przewidującego wyniki meczów piłkarskich z ligi angielskiej, wymaga dostępu do dużej ilości danych danych znajdujących się w wielu repozytoriach. Z uwagi na wspomnianą popularność tej dyscypliny oraz zainteresowanie nią analityków, dostęp do tych danych nie stanowi problemu.

\section{Bazy danych i API}
Podstawą uczenia maszynowego są dane na których algorytmy mogą się uczyć, dlatego jednym z postawionych problemów było miejsce przechowywania danych jak i sposób ich pobierania.

Relacyjne bazy danych (\english{Relational databases, RBD}), których postulaty pojawiły się po raz pierwszy w artykule Edwarda Franka Codd'a \cite{Codd}, są podstawą dla wielu systemów informatycznych, mimo zwiększającej się popularności innych metod przechowywania danych, takich jak NoSQL \cite{RBD_popularity_2016}. Również w obszarze uczenia maszynowego można zauważyć dominującą popularność systemów relacyjnych baz danych. Według ankiety przeprowadzonej przez \textit{Kaggle.com}, trzy najczęściej wybierane bazy danych to relacyjne bazy danych, takie jak \textit{Microsoft SQL Server} \cite{RDB_popularity_kaggle_2020}.

RBD jest zbiorem tabel, z których każda ma za zadanie reprezentować różne encje (\english{entity}), ma przedstawiać część rzeczywistości. W tabelach występują rzędy oraz kolumny. Rzędy to instancje, które przedstawiają konkretne wystąpienie danej encji, a kolumny to ich atrybuty, które reprezentują cechy opisywanej części rzeczywistości. 

\begin{figure}[h] 
        \centering\includegraphics[width=14cm,height=10cm]{figures/Example_entities.PNG}
        \caption{Przykładowych schemat relacji encji relacyjnej bazy danych.}\label{example-Entity}
\end{figure}

Powodem dla którego nazywamy to relacyjną bazą danych jest możliwość wstawienia odnośnika do kolumny, który wskazuje na rząd w innej tabeli. Patrząc na przykładowy schemat \ref{example-Entity}, można zauważyć, że kolumny \textit{TeamA\_Id} oraz \textit{TeamB\_Id} z tabeli \textit{Mecz}, odnoszą się do wartości w tabeli \textit{Drużyna}. Wartości w kolumnach, odwołujące się do innych tabel nazywamy \textit{kluczami obcymi}. \cite{Relational_Databases_Milan}

Obecnie istnieje duża ilość oprogramowania, umożliwiającego zarządzanie takimi bazami danych, np. \textit{Microsoft SQL Server} czy \textit{PostgreSql}, lecz wszystkie sprowadzają się do standardu ISO \cite{SQL_ISO}. Zdefiniowane są tam jaką RBD powinny mieć strukturę oraz jakie typy danych zawierają. W pracy zostały użyte z względu na łatwość dostępu do narzędzi oraz możliwości utworzenia instancji na chmurze, korzystając z Azure Cloud.

Część Web APi

\section{Wstępne przetwarzanie i wizualizacja danych}

\noindent 

Jednym z kluczowych aspektów mocno wpływających na powodzenie każdego projektu związanego z uczeniem maszynowym jest stworzenie odpowiedniego zbioru cech (\english{features}) na podstawie poprawnie przygotowanych danych. W kontekście uczenia maszynowego, cecha to indywidualna, mierzalna własność lub charakterystyka pewnego obserwowanego zjawiska~\cite{Wiki:Feature}. W przypadku modelu przewidującego wyniki meczów piłkarskich, przykładem cechy może być liczba żółtych kartek uzyskanych przez drużynę gospodarzy w ostatnim meczu lub średni procent posiadania piłki drużyny gości w meczach obecnego sezonu.

Metody zbioru danych są bardzo często automatyzowane i pozbawione ścisłej kontroli, stąd mogą one zawierać różnego rodzaju błędy, takie jak wartości spoza zakresu (np. wiek: -20), niemożliwe kombinacje wartości (płeć: mężczyzna, w ciąży: tak) lub pominięte atrybuty dla niektórych rekordów. Ważne jest, aby takie błędy wychwycić już na etapie wstępnego przetwarzania i nie dopuścić ich do danych wejściowych algorytmów. Pomocna może się tutaj okazać wizualizacja dokonana w celu bliższego zapoznania się z danymi. Dla atrybutów można wizualizować rozkład ich wartości przy pomocy histogramów lub wypisywać ich podstawowe statystyki (m.in. średnie, mediany, wartości minimalne, maksymalne) w celu weryfikacji poprawności ich zakresów i typów.

System będzie w stanie nauczyć się przewidywać wyniki tylko mając do dyspozycji jak najwięcej znaczących cech i jak najmniej tych mało znaczących. Zgodnie z popularnym angielskim zwrotem w obszarze przetwarzania informacji-  \definicja{śmieci na wejściu – śmieci na wyjściu} (\english{Garbage In, Garbage Out, GIGO}), nawet skuteczny i poprawnie działający program w przypadku otrzymania na wejściu błędnych danych, da na wyjściu niepoprawne oraz mało użyteczne wyniki, stąd tak ważne jest uprzednie przygotowanie danych.

Wstępne przetwarzanie (\english{preprocessing}) w naszym systemie składa się z jednorazowego dokonania wizualizacji posiadanych przez nas danych (która będzie szczegółowo przedstawiona w późniejszym rozdziale) w celu pogłębienia wiedzy na ich temat i wyszukania potencjalnych braków lub błędów. Następnym zadaniem, wykonywanym każdorazowo przy tworzeniu zbioru danych, którego przeznaczeniem jest użycie podczas testowania różnego rodzaju algorytmów, jest pobranie interesujących danych z bazy, przetworzenie ich, czyli poradzenie sobie z m.in. wartościami pustymi i stworzenie na ich podstawie wektora cech dla każdego rekordu (w przypadku naszego systemu jest to pojedynczy mecz). 

Wynikiem etapu wstępnego przetwarzania jest tzw. zbiór treningowy (\english{training set}), którego znaczenie zostanie przybliżone w podrozdziale traktującym o algorytmach uczenia maszynowego.

\section{Sposób oceny wyników predykcji}
\label{section:ocenaWynikow}

Różne algorytmy dają wyniki w różnych formatach. Przy pomocy jednych uzyskujemy tylko konkretną klasę, do której przynależy konkretne wejście, a inne przyporządkowują rozkład prawdopodobieństwa możliwych wyników klasyfikacji. W naszym przypadku, mamy do czynienia z obiema sytuacjami. Podczas doboru sposobu oceny naszych algorytmów głównym celem było dobranie takich miar, aby móc wzajemnie porównywać zastosowane algorytmy i na tej podstawie dobrać najlepsze podejście. Ocena jakości naszych rozwiązań opierała się zatem na policzeniu \definicja{dokładności} (\english{accuracy}) na zbiorze testowym (wynoszącym 10\% całego zbioru danych), macierzy pomyłek oraz wynikające z niej miary takie jak: \definicja{precyzja} (\english{precision}, interpretujemy jako stosunek tp / (tp + fp), gdzie tp to liczba poprawnie sklasyfikowanych przykładów, a fp to liczba niepoprawnie sklasyfikowanych przykładów.), \definicja{czułość} (\english{recall}, interpretujemy jako stosunek tp / (tp + fn), gdzie tp jest liczbą poprawnie sklasyfikowanych przykładów, a fn liczbą niepoprawnie sklasyfikowanych przykładów.) oraz \definicja{F-score} (\definicja{F-score} interpretuje się jako ważoną średnią harmoniczną precyzji i czułości, gdzie wynik \definicja{F-score} osiąga najwyższą wartość przy 1, a najniższą przy 0), którą wyznaczyliśmy przy użyciu średniej typu „macro” (oblicza metryki dla każdej etykiety, a następnie wyznacza ich nieważoną średnią. Metoda ta nie uwzględnia niezbalansowania danych) \cite{SKfscore}. Dodatkowo do porównania zastosowaliśmy macierz pomyłek, której schemat można zobaczyć w tabeli \ref{tab:macierz}

\begin{center}
\begin{table}
\renewcommand{\arraystretch}{1.5}
\caption{Macierz pomyłek dla naszego problemu klasyfikacji}
\label{tab:macierz}
\begin{center}
\begin{tabular}{|c|c|c|c|c|}
   \cline{3-5} 
   \multicolumn{1}{c}{} & & \multicolumn{3}{c|}{Predicted} \\ \cline{3-5}
   \multicolumn{1}{c}{} & & Draw & HomeWin & AwayWin \\ \hline
   
   {Observed}
   & Draw & .. & .. & ..  \\ \cline{2-5}
   & HomeWin & .. & .. & ..  \\ \cline{2-5}
   & AwayWin & .. & .. & .. \\ \hline
\end{tabular}
\end{center}
\end{table}
\end{center}

Dodatkowo przetestowaliśmy nasze algorytmy w sposób, w którym dane wejściowe podzieliliśmy na bloki następujące po sobie. Następnie wyuczyliśmy nasz algorytm na pierwszym bloku i testowaliśmy jego działanie na mniejszym bloku, który w zbiorze danych występował zaraz po bloku uczącym. Kolejno do dotychczasowego zbioru uczącego dołożyliśmy dane z poprzedniego bloku testowego i na podstawie nowego zbioru znów dokonaliśmy uczenia maszynowego by następnie dokonać ewaluacji na kolejnym fragmencie zbioru testowego. Proces ten powtarzaliśmy, aż do wykorzystania wszystkich dostępnych i przygotowanych danych.

\section{Algorytmy uczenia maszynowego}
opisy teoretyczne wykorzystywanych algorytmów
Przegląd literatury naświetlający stan wiedzy na dany temat obejmuje rozdziały pisane na podstawie
literatury, której wykaz zamieszczany jest w części pracy pt.~\emph{Literatura} (lub inaczej \emph{Bibliografia},
\emph{Piśmiennictwo}). W tekście pracy muszą wystąpić odwołania do wszystkich pozycji zamieszczonych w
wykazie literatury. \textbf{Nie należy odnośników do literatury umieszczać w stopce strony.} Student jest
bezwzględnie zobowiązany do wskazywania źródeł pochodzenia informacji przedstawianych w pracy,
dotyczy to również rysunków, tabel, fragmentów kodu źródłowego programów itd. Należy także podać
adresy stron internetowych w przypadku źródeł pochodzących z Internetu. 


\subsection{Metoda Wektorów Wspierających - SVM}
\definicja{Metoda wektorów wspierających - nośnych} (\english{Support Vector Machine, SVM}), jest to nadzorowana technika uczenia maszynowego wykorzystywana do zadań klasyfikacji oraz regresji. Technika ta została opracowana w laboratorium AT\&T Bell poprzez Vladimira Naumovicha Vapnika wraz ze współpracownikami (Boser i in., 1992, Guyon i in., 1993, Vapnik i in., 1997). \cite{Wiki:SVM}. Głównym założeniem tej metody jest wyznaczenie hiperpłaszczyzny, która ma za zadanie rozdzielić przy pomocy maksymalnego marginesu przykłady należące do różnych klas. W przypadku występowania więcej niż dwóch klas, metoda ta wykorzystuję technikę \definicja{OvR} (\english{One-vs-Rest}), która polega na szkoleniu jednego klasyfikatora na klasę, z próbkami z tej klasy jako pozytywne próbki, a inne próbki jako negatywy. Formalnie problem klasyfikacji przedstawia się następująco \cite{Prezentacja:SVM2}: 
\begin{itemize}
    \item W przestrzeni danych (ang. measurement space) $\omega$ znajdują się wektory danych x stanowiące próbkę uczącą D, należące do dwóch klas\\
    \begin{equation}
D = \big\{(x_{i}, c_{i}) | x_{i} \in R^{p}, c_{i} \in \{-1, 1\}\big\}_{i=1}^{N}
    \end{equation}
    \item Szukamy klasyfikatora pozwalającego na podział całej przestrzeni $\omega$ na dwa rozłączne obszary odpowiadającej klasom {1,-1} oraz pozwalającego jak najlepiej klasyfikować nowe obiekty x do klas
    \item Podejście opiera się na znalezieniu tzw. granicy decyzyjnej między klasami → g( x )
\end{itemize}
Dodatkowo dwie klasy są liniowo separowalne, jeśli istnieje hiperpłaszczyzna H postaci: \[g(x) = w^Tx + b\] przyjmująca wartości: 

\[
    \begin{cases}
            g(x_{i}) > 0,& x_{i} \in 1 \\
            g(x_{i}) < 0,& x_{i} \in -1
    \end{cases}
\]
\\

Linia granicy decyzyjnej nie tylko oddziela dwie klasy, ale także pozostaje jak najdalej od najbliższych instancji. Jak pokazano na rysunku \ref{SVM-margines} problemem poza znalezieniem $B_{2}$ jest również znalezienie szerokości wspomnianej granicy. Niestety nie ma na to idealnego rozwiązania i konieczne jest wybranie cech, które są bardziej odpowiednie dla naszego problemu. Szerszy margines to lepsze własności generalizacji, mniejsza podatność na
ewentualne przeuczenie (\english{overfitting}), a z kolei wykorzystanie wąskiego marginesu skutkuje radykalną zmianą klasyfikacji przy małej zmianie granicy. Jednak ze względu na oszacowanie górnej granicy błędu ze względu na błąd uczący częściej wybiera się jak najszerszy margines w celu lepszego uogólniania w bardziej skomplikowanych modelach.

\begin{figure}[h] 
        \centering\includegraphics[width=6cm,height=6cm]{figures/SVM-margin.png}
        \caption{Margines granicy decyzyjnej}\label{SVM-margines}
\end{figure}
Konsekwentnie, problem maszyn wektorów nośnych sprowadza się do poszukiwania maksymalnego marginesu klasyfikacji: $w x + b = 0$ gdzie \definicja{w} oraz \definicja{b} są parametrami modelu.
\[
y = 
    \begin{cases}
            1,&  wx+b > 0\\
            -1,& wx+b < 0
    \end{cases}
\]
Dodatkowo parametry granicy wyznacza się tak, aby maksymalne marginesy (margines to odległość między \definicja{bi1} oraz \definicja{bi2}) \definicja{bi1} i \definicja{bi2} były miejscem geometrycznym punktów \definicja{x} spełniających warunki \ref{SVM-marginesEq}:
\[
    \begin{cases}
            bi1,&  wx+b = 1\\
            bi2,& wx+b= -1
    \end{cases}
\]
\begin{figure}[h] 
        \centering\includegraphics[width=14cm,height=6cm]{figures/SVM-marginEq.png}
        \caption{Wyznaczanie marginesu granicy decyzyjnej}\label{SVM-marginesEq}
\end{figure}
\\
Tak więc problem ten sprowadza się do optymalizacji kwadratowej z liniowymi ograniczeniami (uogólnione zadanie optymalizacji). 
Czasami jednak mamy do czynienia z sytuacją podczas której nie mamy możliwości w pełni liniowej separacji klas. W takiej sytuacji wykorzystuje się zmienne osłabiające \definicja{$\xi_{i} \ge 0$}, które dobiera się dla każdego przykładu uczącego. Jej wartość zmniejsza margines separacji. Jeżeli $0 \le \xi_{i} \le 1$, to punkt danych $(xi,di)$ leży wewnątrz strefy separacji, ale po właściwej stronie, a w sytuacji gdy $\xi_{i} \ge 1$, to punkt leży po niewłaściwej stronie hiperpłaszczyzny i nastąpi błąd klasyfikacji. 
\[
    \begin{cases}
            bi1,&  wx+b = 1 - \xi\\
            bi2,& wx+b= -1 + \xi
    \end{cases}
\]
Również tutaj występuje konflikt doboru marginesu. Szeroki margines to dużo błędów i odwrotnie. Kończąc rozważania dotyczące liniowej maszyny wektorów nośnych, nasz problem znalezienia granicy decyzyjnej sprowadza się do minimalizacji wyrażenia:
\[
L(w) = \frac{\|\vec{w}\|}{2} + C\big(\sum_{i=1}^{N}\xi_{i}^{k}\big)
\]
z ograniczeniami:
\[
f(\vec{x_{i}}) = 
    \begin{cases}
            1 &  \text{if}\ \vec{w} \bullet \vec{x_{i}}+b \ge 1 - \xi_{i}\\
            -1 &  \text{if}\ \vec{w} \bullet \vec{x_{i}}+b \le 1 + \xi_{i}
    \end{cases}
\]
gdzie parametr \definicja{C} to ocena straty związanej z każdym błędnie klasyfikowanym punktem dla którego $\xi > 0$. Problem ten to problem dualny i istnieją techniki pozwalające na jego rozwiązanie.

Dodatkowym problemem w tej metodzie jest fakt, że najczęściej klasy nie są liniowo separowane. Jednym ze sposobów na poradzenie sobie z tym problemem jest transformowanie, projekcja danych wejściowych do przestrzeni o większej liczbie wymiarów, w której dane, z dużym prawdopodobieństwem będą separowane liniowo. Funkcja decyzyjna po przekształceniu ma się następująco:
\[
g(x) = w\varphi(x) + b
\]
Problem ten jest trudny obliczeniowo do wykonania, lecz można sobie z nim poradzić za pomocą kerneli, funkcji jądrowych. Funkcje te wywodzą się z badań liniowych przestrzeni wektorowych, przestrzeni Hilberta, Banacha. Dzięki nim można wyznaczyć potrzebne parametry do rozwiązania naszego problemu. Najczęściej stosowane jądra w metodzie maszyn wektorów nośnych to: Gaussowskie, wielomianowe i sigmoidalne. Dzięki nim, nie musimy znać funkcji transformacji, a jedynie funkcję kernela co pozwala nam na pracę w nowej przestrzeni. 

Można zauważyć, że metoda SVM jest silną metodą pozwalającą na rozwiązywanie problemów klasyfikacji w wielowymiarowej przestrzeni danych. Dzięki niej można skutecznie uogólniać nowe przykłady i przydzielać im odpowiednie klasy zdefiniowane dla konkretnego problemu.

\subsection{Sztuczne Sieci Neuronowe - SNN}

\definicja{Sztuczna sieć neuronowa} ogólna nazwa struktur matematycznych i ich programowych lub sprzętowych modeli, realizujących obliczenia lub przetwarzanie sygnałów poprzez rzędy elementów przetwarzających, zwanych sztucznymi neuronami, wykonujących pewną podstawową operację na swoim wejściu. Oryginalną inspiracją takiej struktury była budowa naturalnych neuronów, łączących je synaps, oraz układów nerwowych, w szczególności mózgu \cite{Wiki:SNN}. Sztuczne sieci neuronowe charakteryzują się tym, że mają zdolność do odwzorowania różnych zależności pomiędzy sygnałami wejściowymi i wyjściowymi. Sztuczna sieć neuronowa składa się z wielu połączonych neuronów, a kady neuron można interpretować jako pewna kombinacja matematyczna cech wejściowych wraz z przypisanymi im wagami. Wyjście neuronu to pewna funkcja matematyczna, która przekształca daną kombinację danych wejściowych i przekazuje taki wynik na swoje wyjście. Graficzna reprezentacja takiego neuronu ma się następująco \cite{Prezentacja:SNN} --- \textbf{to jest także ze skryptu KKJS}:
\begin{figure}[h] 
        \centering\includegraphics[width=12cm,height=6cm]{figures/SNN.png}
        \caption{Sztuczny neuron}\label{SVM-neuron}
\end{figure}

Podstawowe elementy składowe: 
\begin{itemize}
    \item n wejść neuronu wraz z wagami $w_{i}$ (wektor wag w i wektor sygnałów wejściowych x)
    \item jeden sygnał wyjściowy y
    \item pobudzenie e neuronu jako suma ważona sygnałów wejściowych
pomniejszona o próg $\Theta$
\[
e = \sum_{i=1}^{N} w_{i} x_{i} - \Theta = w^{T} x - \Theta
\]
wprowadźmy wagę $w_{0}= \Theta$, podłączonej do stałego sygnału $x_{0} = 1$;
wówczas: 
\[
e = \sum_{i=0}^{N} w_{i} x_{i}  = w^{T} x
\]
\item funkcja aktywacji (przejścia):
\[
y = f(e)
\]
\end{itemize}
Kluczowe znaczenie dla działania neuronu ma funkcja aktywacji. Mamy do wyboru liniową funkcję lub nielinową (ciągła i nieciągła, unipolarna i bipolarna).

W literaturze wyróżnia się dwa ogólne typy sieci neuronowych i są to: jednokierunkowe SNN oraz rekurencjne SNN (sieci ze sprzężeniami zwrotnymi np. sieć Hopfielda albo sieci uczenia się przez współzawodnictwo).
Sieci jednokierunkowe to sieci o jednym kierunku przepływu sygnałów. Szczególnym przypadkiem architektury jednokierunkowej jest sieć warstwowa, reprezentująca zdecydowanie najpopularniejszą topologię.
\begin{figure}[h] 
        \centering\includegraphics[width=12cm,height=6cm]{figures/ArchitekturaSNN.png}
        \caption{Sieć warstwowa jednokierunkowa}\label{SVM-neuron}
\end{figure}

Zasady łączenia neuronów między sobą:
\begin{itemize}
    \item każdy neuron z każdym,
    \item połączenia między kolejnymi warstwami w sieciach warstwowych,
    \item tylko z pewną grupą neuronów, najczęściej z tzw. sąsiedztwem
\end{itemize}

W celu uzyskania wyników wykorzystuje się \definicja{uczenie nadzorowane}. Dany jest zbiór przykładów uczących składający się z par wejście-wyjście $(x_{j}, z_{j})$, gdzie $z_{j}$ jest pożądaną odpowiedzią sieci na sygnały wejściowe $x_{j} (j=1,..m)$. Zadaniem sieci jest nauczyć się możliwie jak najdokładniej funkcji przybliżającej powiązanie wejścia z wyjściem. Odległość pomiędzy rzeczywistą a pożądaną odpowiedzią sieci jest
miarą błędu używaną do korekcji wag sieci. Typowym przykładem jest uczenie sieci wielowarstwowej algorytmem wstecznej propagacji błędu; każdy neuron lokalnie zmniejsza swój błąd stosując metodę spadku gradientu.

W celu uczenia sieci neuronowych wykorzystuje się kilka reguł i są nimi między innymi: \definicja{reguła Widrowa-Hoffa} oraz \definicja{reguła delta}. Pierwsza z nich dotyczy uczenia nadzorowanego sieci jednokierunkowych, gdzie minimalizuje się błąd pomiędzy pożądaną a aktualną odpowiedzią. 
\[
\delta^{j} = z^{j} - y^{j} = z^{j} - w^{T}x^{j}
\]

Korekta wag jest następująca \cite{Widrow}:
\begin{equation}
\label{eqn:delta}
\delta w_{i} = \eta \delta^{j} x^{j}_{i}
\end{equation}
Reguła delta z kolei obowiązuje dla neuronów z ciągłymi funkcjami aktywacji i nadzorowanego trybu uczenia. Regułę delta wyprowadza się jako wynik minimalizacji kryterium błędu średnio-kwadratowego Q.
\[
Q = \frac{1}{2}\sum_{j=1}^{N} \big( z^{j} - y^{j}\big)^{2} = \sum_{j}^{N}Q^{j}, Q^{j} = \frac{1}{2}(\delta^{j})^{2}
\]

Korekta wag:
\[
\delta w_{i} = \eta \delta^{j} (1 - y^{j}) f'(e^{j}) x^{j}_{i}
\]
gdzie $f'()$ oznacza pochodną funkcji aktywacji. Stosowana jest do uczenia wielowarstwowych sieci neuronowych wraz z algorytmem wstecznej propagacji błędów \cite{Rumelhart}. Sieci charakterystyczne cechują się pewnymi właściwościami \cite{Mitchell}:
\begin{itemize}
    \item Przykłady uczące opisane są przez pary atrybut-wartość (na ogół
zdefiniowanych na skalach liczbowych,
    \item Przybliżana funkcja może mieć wartości dyskretne lub rzeczywiste;
może być także wektorem wartości,
 \item Dane mogą zawierać błędy lub podlegać zniekształceniu. SSN są
odporne na różnego rodzaju uszkodzenia danych, 
    \item Akceptowalny jest długi czas uczenia sieci,
    \item Akceptacja dla potencjalnie dużej liczby parametrów algorytmu,
które wymagają dostrojenia metodami eksperymentalnymi,
    \item Zadanie nie wymaga rozumienia przez człowieka funkcji
nauczonej przez SNN - trudności z interpretacją wiedzy nabytej
przez sieć (rozwijający się w tym momencie sektor uczenia maszynowego XAI - \english{explainable artificial intelligence}).
\end{itemize}

Jednak kluczowym konceptem sprawiającym, że sztuczne sieci neuronowe są tak popularne i oferujące wiele możliwości jest wsteczna propagacja błędów, czyli sposób w jaki sieć nabywa umiejętności generalizowania danych i przyporządkowywania odpowiednich wartości. Błąd k-tego neuronu w l-tej warstwie jest równy sumie błędów popełnionych przez neurony (p) z warstwy l+1-szej ważonych po wagach $w_{k(p,l+1)}$ łączących ten neuron z neuronami tej warstwy \cite{Prezentacja:SNN}: 

\[
\delta_{(k,l)}^{j} = \sum_{p=1}^{N_{l+1}} W^{j}_{k(p, l+1)} \delta_{(p,l+1)}^{j}
\]

Na podstawie wstecznej propagacji błędów, sieci neuronowe modyfikują wagi dla każdego neuronu dopowiadające danym wejściowym w celu minimalizacji funkcji nazywanej \definicja{funkcją straty}.

Wartości początkowe wag muszą być zainicjowane przed przystąpieniem do procesu uczenia i zazwyczaj dokonuje się tego w sposób losowy lub na podstawie pewnego rozkładu prawdopodobieństwa. Kluczowym czynnikiem do prędkości oraz jakości otrzymywanych wyników jest współczynnik $\eta$, który już mogliśmy zauważyć w równaniu \ref{eqn:delta}. Decyduje on o wpływie błędu popełnianego przez neuron na korektę wartości wag. Właściwy dobór ma kluczowe znaczenie dla prędkości zbieżności algorytmu, a jego zbyt mała wartość spowalnia proces uczenia i zwiększa ryzyko
wpadnięcia w pułapkę lokalnego minimum \cite{Prezentacja:SNN}. 

Niestety sieć neuronowa jest trudnym narzędziem i występuję w niej problem doboru wielkości warstw ukrytych, który do teraz jest problemem otwartym. 
Nie istnieje jednoznaczna reguła określająca optymalny rozmiar danej warstwy oraz ilości warstw w danej sieci przy danym zbiorze uczącym. Zbyt mała wielkość warstw czyni sieć niezdolną do adaptacji do
zadanego zbioru przykładów co skutkuje, że w trakcie uczenia błąd
średnio-kwadratowy utrzymuje dużą wartość. Zbyt duże warstwy z kolei mają problem z przeuczaniem (nauka konkretnych wartości danych wejściowych, a nie ogólnego konceptu, zarysu) \cite{Prezentacja:SNN}.

Od początku historii SNN do teraz powstało mnóstwo konceptów, pomysłów, sztuczek i schematów które zapewniają lepsze generalizowanie, szybsze zbieganie do minimum, które nie sposób zebrać i opisać w jednym miejscu. Wiele technik zapewniło uczenie bardzo dużych sieci neuronowych zapewniających stabilne gradienty (jednym z problemów podczas nauki SNN jest niestabilność gradientów, które zanikały wraz z postępem algorytmu lub wręcz eksplodowały do bardzo dużych wartości) \cite{gradient}. Powstało również wiele funkcji aktywacji oraz sposobów inicjalizacji początkowych wag sieci.


\chapter{Architektura}
\section{Schemat systemu}
\section{Schemat bazy danych}
\section{Opis web API}
\section{Środowisko użytkownika docelowego}
\chapter{Własne propozycje rozwiązań}
    \section{Pozyskanie i agregowanie danych} \label{data_aggregation}
        \subsection{Europejska baza danych piłkarskich}
        \subsection{Elo Rating klubów piłkarskich}
        \subsection{Historyczne dane zakładów piłkarskich}
        \subsection{Metody agregacji}
    \section{Web API}
    \section{Wizualizacja charakterystyk danych}
    \noindent Przed przystąpieniem do właściwego przetwarzania danych i tworzenia na ich podstawie cech, warto bliżej im się przyjrzeć w celu potwierdzenia ich prawidłowości i wyszukania potencjalnych błędów i braków. Ten podrozdział poświęcony jest wizualizacji charakterystyk danych, które posiadamy w naszej bazie. Wizualizacja została dokonana w \emph{Jupyter Notebook}, a jej plik źródłowy znajduje się w repozytorium projektu~\cite{repo} w folderze \texttt{visualization/} pod nazwą \texttt{visualization.ipynb}
    
    Podczas wizualizacji pomijane są tabele, które są z jej punktu widzenia mało istotne - tabelę \emph{Team} (ponieważ zawiera ona tylko nazwy drużyn) i tabele \emph{Country} oraz \emph{League} (posiadają one tylko jeden wiersz - analizujemy rozgrywki z jednego kraju oraz jednej ligi). Podobnie z atrybutami - pomijane będą klucze podstawowe oraz klucze obce.
    
    ~
    
    Pierwszą tabelą wartą głębszej analizy jest tabela \emph{Team\_Attributes}. Zawiera ona pewne statystyki dla każdej z drużyn. Oto jak prezentuje się zestawienie kolumn tej tabeli wraz z ich typami i liczbą wartości niepustych:
    
    \begin{table}[H]
    \caption{Kolumny tabeli Team\_Attributes}
    \centering\footnotesize%
    \begin{tabular}{l c c}
    \toprule
        Nazwa & Liczba niepustych wartości & Typ \\
    \midrule
        date & 204 & datetime64 \\
        buildUpPlaySpeed & 204 & int64 \\
        buildUpPlaySpeedClass & 204 & int64 \\
        buildUpPlayDribbling & 68 & float64 \\
        buildUpPlayDribblingClass & 204 & string \\
        buildUpPlayPassing & 204 & float64 \\
        buildUpPlayPassingClass & 204 & string \\
        buildUpPlayPositioningClass & 204 & string \\
        chanceCreationPassing & 204 & int64 \\
        chanceCreationPassingClass & 204 & string \\
        chanceCreationCrossing & 204 & int64 \\ 
        chanceCreationCrossingClass & 204 & string \\        
        chanceCreationShooting & 204 & int64 \\  
        chanceCreationShootingClass & 204 & string \\
        chanceCreationPositioningClass & 204 & string \\
        defencePressure & 204 & int64 \\
        defencePressureClass & 204 & string \\
        defenceAggression & 204 & int64 \\  
        defenceAggressionClass & 204 & string \\
        defenceTeamWidth & 204 & int64 \\
        defenceTeamWidthClass & 204 & string \\
        defenceDefenderLineClass & 204 & string \\
    \bottomrule
    \end{tabular}
    \end{table}
    
    \noindent Wszystkich wierszy w tabeli jest 204, stąd na powyższym zestawieniu widać, że jedynym atrybutem, który nie ma uzupełnionych wartości dla wszystkich rekordów jest \emph{buildUpPlayDribbling}. Z uwagi na to, że brakujących wartości jest dosyć dużo (prawie 70\% wierszy w tabeli nie ma podanej wartości dla tego atrybutu), zdecydowano się go nie uwzględniać przy dalszym przetwarzaniu. Pomijane są również wszystkie kolumny typu \emph{string}, ponieważ są one tylko kategoriami (klasami) dla odpowiadających im wartości numerycznych.
    
    Warto również przyjrzeć się histogramom wszystkich atrybutów numerycznych w celu weryfikacji poprawności ich zakresów.
    
    \begin{figure}[H] 
        \centering\includegraphics[width=\textwidth]{figures/team_attributes.png}
        \caption{Histogramy wartości numerycznych tabeli Team\_Attributes}
        \label{fig:team_attributes}
    \end{figure}
    
    \noindent Jak można zaobserwować na rysunku~\ref{fig:team_attributes}, wszystkie atrybuty mają odpowiednie zakresy - nie ma wartości odstających, wszystkie są dodatnie. W ich przypadku dalsze przetwarzanie nie będzie konieczne.\\*
    
    \noindent Kolejną analizowaną tabelą jest tabela \emph{Player}. Posiada ona podstawowe informacje o wszystkich graczach.
    
    \begin{table}[H]
    \caption{Kolumny tabeli \emph{Player}}\label{tab:player}
    \centering\footnotesize%
    \begin{tabular}{l c c}
    \toprule
        Nazwa & Liczba niepustych wartości & Typ \\
    \midrule
        player\_name & 1397 & string \\
        birthday & 1397 & datetime64 \\
        height & 1397 & int64 \\
        weight & 1397 & int64 \\
    \bottomrule
    \end{tabular}
    \end{table}
    
    \noindent Wszystkich wierszy w tabeli Player jest 1397. Zgodnie z tablicą~\ref{tab:player}, tabela nie posiada wartości pustych. Z uwagi na trudność wizualizacji atrybutu \emph{birthday}, wprowadzony zostanie nowy, sztuczny atrybut wieku (\english{age}), który pozwoli zweryfikować jego poprawność. \\*
    
    \begin{figure}[H] 
        \centering\includegraphics[width=\textwidth]{figures/player.png}
        \caption{Histogramy wartości dla atrybutów tabeli \emph{Player}}
        \label{fig:player}
    \end{figure}
    
    \noindent Spoglądając na rysunek~\ref{fig:player} zauważamy, że rozkłady wartości wszystkich atrybutów przypominają rozkłady normalne. Tutaj podobnie jak w przypadku poprzedniej tabeli, zakresy są poprawne. Warto zaznaczyć, że waga (\english{weight}) na potrzeby wizualizacji została przedstawiona w kilogramach, a wysokość (\english{height}) w centymetrach. Omówiony wcześniej sztuczny atrybut wieku pozwolił zweryfikować prawidłowość wartości dla kolumny \textit{birthday}. \\*
    
    \noindent Inną tabelą związaną z graczami jest tabela \emph{Player\_Attributes}. Zawiera ona pewne statystyki odnośnie każdego z graczy dotyczące jego umiejętności. Przyglądając się schematowi naszej bazy danych na rysunku~\ref{database_schema}, można zaobserwować, że tabela ta posiada dużo różnych atrybutów numerycznych. Istotne jednak z punktu widzenia późniejszego tworzenia cech jest istnienie atrybutu zwanego \emph{overall\_rating}, który można na język polski przetłumaczyć jako ,,ogólna ocena''. Jest on pewną agregacją wszystkich pozostałych atrybutów dokonaną przez twórców gry Fifa. Dzięki niemu można pominąć pozostałe atrybuty i do dalszego przetwarzania uwzględnić wyłącznie ten.
    
    \begin{figure}[H] 
        \centering\includegraphics[width=5cm]{figures/overall_rating.png}
        \caption{Histogram wartości atrybutu \emph{overall\_rating}}
        \label{fig:overall_rating}
    \end{figure}
    
    \noindent Dodatkowym jego atutem (co można zauważyć na rysunku~\ref{fig:overall_rating}) jest poprawność zakresu jego wartości oraz fakt, że jest on zdefiniowany dla każdego gracza w bazie danych.
    
    Atrybuty kategoryczne, m.in \emph{preferred foot} (,,preferowana noga''), zostały pominięte ze względu na wątpliwy wpływ na predykcję wyniku meczu. \\*
    
    \noindent \emph{EloRating} jest tabelą, która zawiera wartości tzw. atrybutu \emph{Elo} aktualnego na dany dzień dla każdej z drużyn.
    
    \begin{table}[H]
    \caption{Kolumny tabeli EloRating}\label{tab:elo}
    \centering\footnotesize%
    \begin{tabular}{l c c}
    \toprule
        Nazwa & Liczba niepustych wartości & Typ \\
    \midrule
        team\_name & 31607 & string \\
        rank & 31607 & int64 \\
        Elo & 31607 & float64 \\
        start\_date & 31607 & datetime64 \\
        end\_date & 31607 & datetime64 \\
    \bottomrule
    \end{tabular}
    \end{table}
    
    \begin{table}[H]
    \caption{Przykładowe rekordy w tabeli EloRating}\label{tab:elo_example}
    \centering\footnotesize%
    \begin{tabular}{l c c c c}
    \toprule
        team\_name & rank & Elo & start\_date & end\_date \\
    \midrule
        Manchester United & 14 & 1843.027344 & 2014-08-17 & 2014-08-19\\
        Manchester United & 14 & 1842.388550 & 2014-08-20 & 2014-08-21\\
        Manchester United & 15 & 1840.082642 & 2014-08-22 & 2014-08-23\\
        Manchester United & 14 & 1840.082642 & 2014-08-24 & 2014-08-24\\
        Manchester United & 14 & 1836.078979 & 2014-08-25 & 2014-08-27\\
        Manchester United & 15 & 1837.623047 & 2014-08-28 & 2014-08-28\\
    \bottomrule
    \end{tabular}
    \end{table}
    
    \noindent Tablica~\ref{tab:elo} stanowi potwierdzenie uzupełnionych wartości atrybutów dla każdego rekordu w tabeli, a tablica~\ref{tab:elo_example} obrazuje, że dla każdej drużyny w naszej bazie może istnieć wiele rekordów. Każdy zawiera inną wartość atrybutu \emph{Elo} (wraz z miejscem zajmowanym w globalnym rankingu drużyn) aktualnym w danym przedziale czasowym. Atrybut ten ulega przeliczeniu po każdym rozegranym przez drużynę meczu.
    
    Tworząc wektory cech reprezentujące pojedynczy mecz, dla obu drużyn korzystamy z wartości tego atrybutu aktualnego na dzień meczu. Obrazuje on aktualną ,,formę'' drużyny w dniu rozgrywania meczu. \\*
    
    \noindent Najobszerniejszą tabelą w naszym systemie jest tabela \emph{Matches}. Każdy jej rekord reprezentuje jeden mecz pomiędzy dwoma drużynami wraz z jego charakterystykami, np. liczba strzałów drużyny gospodarzy, liczba żółtych kartek drużyny gości, kursy oferowane na remis itd.
    
    Z uwagi na sporą liczbę atrybutów w tej tabeli, warto ją podzielić na mniejsze części w celu łatwiejszej i czytelniejszej analizy. Pierwszą badanym zestawem są atrybuty związane z kursami od różnych zakładów bukmacherskich.
    
    \begin{table}[H]
    \caption{Podstawowe statystyki dla atrybutów związanych z kursami.}\label{tab:odds}
    \centering\footnotesize%
    \begin{tabular}{l c c c c c}
    \toprule
        Atrybut & \emph{count} & \emph{mean} & \emph{std} & \emph{min} & \emph{max} \\
    \midrule
        B365H & 3040 & 2.701964 & 1.689834 & 1.100000 & 15.000000\\
        BWH & 3039 & 2.603906 & 1.528913 & 1.100000 & 12.500000\\
        IWH & 3038 & 2.513644 & 1.368492 & 1.050000 & 10.000000\\
        LBH & 3039 & 2.611787 & 1.526034 & 1.080000 & 12.000000\\
        PSH & 1519 & 2.720369 & 1.575303 & 1.130000 & 10.800000\\
        WHH & 3040 & 2.663760 & 1.590124 & 1.100000 & 12.000000\\
        SJH & 2320 & 2.667828 & 1.698067 & 1.110000 & 15.000000\\
        VCH & 3040 & 2.712204 & 1.692001 & 1.090000 & 15.000000\\
        GBH & 1899 & 2.606840 & 1.586001 & 1.100000 & 12.000000\\
        BSH & 1900 & 2.625553 & 1.655792 & 1.100000 & 13.000000\\\\
        
        B365D & 3040 & 3.952720 & 0.998305 & 3.000000 & 11.000000\\
        BWD & 3039 & 3.803251 & 0.882597 & 2.900000 & 9.250000\\
        IWD & 3038 & 3.687903 & 0.735121 & 3.000000 & 10.000000\\
        LBD & 3039 & 3.815271 & 0.876612 & 2.880000 & 10.000000\\
        PSD & 1519 & 4.078183 & 1.046468 & 3.040000 & 11.030000\\
        WHD & 3040 & 3.669260 & 0.812509 & 2.800000 & 9.500000\\
        SJD & 2320 & 3.881039 & 0.926573 & 3.000000 & 9.000000\\
        VCD & 3040 & 3.955470 & 1.009081 & 2.500000 & 10.000000\\
        GBD & 1899 & 3.761980 & 0.842088 & 3.000000 & 8.500000\\
        BSD & 1900 & 0.837965 & 0.837965 & 3.000000 & 8.500000\\\\
        
        B365A & 3040 & 4.910437 & 3.909392 & 1.220000 & 29.000000\\
        BWA & 3039 & 4.495166 & 3.213833 & 1.220000 & 21.000000\\
        IWA & 3038 & 4.245671 & 2.919038 & 1.270000 & 25.000000\\
        LBA & 3039 & 4.563537 & 3.371435 & 1.220000 & 26.000000\\
        PSA & 1519 & 4.923970 & 3.840303 & 1.370000 & 28.500000\\
        WHA & 3040 & 4.699592 & 3.628486 & 1.220000 & 26.000000\\
        SJA & 2320 & 4.859233 & 3.826951 & 1.250000 & 29.000000\\
        VCA & 3040 & 4.956711 & 3.984335 & 1.220000 & 29.000000\\
        GBA & 1899 & 4.566435 & 3.301710 & 1.250000 & 21.000000\\
        BSA & 1900 & 3.786069 & 3.786069 & 1.220000 & 26.000000\\
    \bottomrule
    \end{tabular}
    \end{table}
    
    \noindent Tablica~\ref{tab:odds} zawiera pogrupowane kursy od różnych bukmacherów według zdarzeń. Ostatnia litera w nazwie atrybutu informuje o typie zdarzenia, na który jest oferowany ten kurs. Litera ,,H'' oznacza kurs oferowany na zwycięstwo drużyny gospodarzy, ,,D'' na remis, natomiast ,,A'' na zwycięstwo gości. Pozostałe litery tworzą skrót od nazwy danego bukmachera. Prezentowane statystyki to liczność (\english{count}) średnia (\english{mean}), odchylenie standardowe (\english{standard deviaton, std}), wartość minimalna oraz maksymalna.
    
    Jak można spostrzec analizując wartości średnie dla poszczególnych grup, na zwycięstwo gospodarzy oferowany jest średnio mniejszy kurs niż na pozostałe dwa zdarzenia. Pozwala to podejrzewać, że granie na swoim stadionie daje drużynie pewną przewagę. Potwierdza to fakt, że na zwycięstwo gości średnio ten kurs jest najwyższy.
    
    Warto również zwrócić uwagę na to, że w obrębie poszczególnych grup zdarzeń, średnie kursy proponowane przez bukmacherów są bardzo do siebie zbliżone. Oznacza to, że najprawdopodobniej ich modele matematyczne wyznaczają podobne prawdopodobieństwa dla danych zdarzeń. Dodatkowym istotnym faktem jest to, że dla każdej grupy zdarzeń istnieją przynajmniej dwa atrybuty, które są zdefiniowane dla każdego meczu (których liczność w naszej bazie danych wynosi 3400). Przykładowo, dla zdarzeń z grupy ,,zwycięstwo gospodarzy'' zawsze istnieje wartość dla atrybutu \emph{B365H} oraz \emph{WHH} (ich wartość \emph{count} wynosi 3400). Ten fakt, wraz z obserwacją o podobnych kursach w obrębie poszczególnych grup, umożliwiają poradzenie sobie z brakującymi wartościami niektórych atrybutów. Skoro wiadomo, że kursy na dane zdarzenie są podobne, wystarczy uwzględniać podczas późniejszego przetwarzania tylko te atrybuty, które mają wartość niepustą (mamy gwarancję obecności przynajmniej dwóch takich atrybutów dla każdego z trzech zdarzeń w meczu- zwycięstwa gospodarzy, remisu oraz zwycięstwa gości). \\*
    
    \noindent Kolejnym zestawem analizowanych atrybutów są te związane bezpośrednio z przebiegiem meczu. 
    
    \begin{table}[H]
    \caption{Wybrane atrybuty tabeli \emph{Matches}}\label{tab:matches}
    \centering\footnotesize%
    \begin{tabular}{l c c l}
    \toprule
        Nazwa & Liczba niepustych wartości & Typ & Wyjaśnienie \\
    \midrule
        HomeTeamShots & 3040 & int64 & liczba strzałów drużyny gospodarzy \\
        AwayTeamShots & 3040 & int64 & liczba strzałów drużyny gości \\
        HomeTeamShotsOnTarget & 3040 & int64 & liczba celnych strzałów drużyny gospodarzy \\
        AwayTeamShotsOnTarget & 3040 & int64 & liczba celnych strzałów drużyny gości \\
        HomeTeamCorners & 3040 & int64 & liczba rzutów rożnych drużyny gospodarzy \\
        AwayTeamCorners & 3040 & int64 & liczba rzutów rożnych drużyny gości \\
        HomeTeamFoulsCommitted & 3040 & int64 & liczba popełnionych fauli przez gospodarzy \\
        AwayTeamFoulsCommitted & 3040 & int64 & liczba popełnionych fauli przez gości \\
        HomeTeamYellowCards & 3040 & int64 & liczba żółtych kartek otrzymanych przez gospodarzy \\
        AwayTeamYellowCards & 3040 & int64 & liczba żółtych kartek otrzymanych przez gości \\
        AwayTeamRedCards & 3040 & int64 & liczba czerwonych kartek otrzymanych przez gości \\
        HomeTeamRedCards & 3040 & int64 & liczba żółtych kartek otrzymanych przez gospodarzy \\
    \bottomrule
    \end{tabular}
    \end{table}

     \begin{figure}[H] 
        \centering\includegraphics[width=\textwidth]{figures/matches.png}
        \caption{Histogramy wybranych atrybutów tabeli \emph{Matches}}
        \label{fig:matches}
    \end{figure}
    
    \noindent Tablica~\ref{tab:matches} utwierdza nas w przekonaniu, iż wszystkie atrybuty są uzupełnione i nie zawierają żadnych braków.
    Na rysunku~\ref{fig:matches} przedstawiono ich histogramy. Potwierdzają one zarówno poprawność ich zakresów, jak i pewne intuicyjne założenia, przykładowo - czerwone kartki są sporadycznym zdarzeniem i występują zdecydowanie rzadziej niż żółte kartki. 
    
    \section{Wstępne przetwarzanie}
    \noindent Na rysunku~\ref{fig:preprocessing_structure} przedstawiona została struktura modułu służącego do pobierania danych z serwera, ich wstępnego przetwarzania oraz tworzenia na ich podstawie cech. Moduł ten znajduje się w repozytorium projektu~\cite{repo} w folderze \texttt{Preprocessing/}.
    \begin{figure}[H] 
        \centering\includegraphics[width=6cm]{figures/preprocessing_structure.png}
        \caption{Struktura modułu do wstępnego przetwarzania}
        \label{fig:preprocessing_structure}
    \end{figure}
    
    Oprócz podfolderu \texttt{test/} (który zostanie szczegółowo omówiony w późniejszym podrozdziale), przechowującego wszystkie testy jednostkowe, znajdują się w nim też właściwe pliki odpowiedzialne za wspomniane funkcje modułu:
    
    \begin{itemize}
        \item \texttt{api.py}- plik zawierający funkcje pomocnicze do tworzenia odpowiedniego url'a, przy pomocy którego następuje wysłanie zapytanie do serwera,
        \item \texttt{data\_fetcher.py}- odpowiada za pobieranie danych z serwera przy użyciu \emph{WebApi},
        \item \texttt{data\_aggregator.py}- używa poprzedniego pliku w celu pobrania danych i służy do ich agregacji (np. danych dla kilku sezonów) w jeden zbiór danych; jest punktem wejściowym do biblioteki, który używa klient,
        \item \texttt{preprocessor.py}- przetwarza otrzymane dane i tworzy na ich podstawie wektory cech reprezentujące każdy mecz,
        \item \texttt{match.py}, \texttt{parameters.py}, \texttt{past\_encounters.py}, \texttt{season.py} oraz \texttt{team.py}- reprezentują modele danych używane w systemie i enkapsulują ich pewne właściwości (np. identyfikator dla sezonu lub drużyny, który jest potrzebny do wykonania zapytania do serwera),
        \item \texttt{main.py}- zawiera przykład użycia biblioteki.
    \end{itemize}
    
        \subsection{Pobieranie danych z serwera i ich agregacja}
        \noindent Moduł komunikuje się z serwerem przy pomocy udostępnionego przez niego interfejsu programistycznego. Komunikacja ta jest inicjowana przez klienta przy użyciu dwóch funkcji z pliku \texttt{data\_aggregator.py}:
        
        \begin{itemize}
            \item \texttt{get\_data\_for\_seasons()}
            \item \texttt{get\_data\_for\_team\_in\_seasons()}
        \end{itemize}
        
        Obie te funkcje przyjmują jako argument listę sezonów, dla których mają być pobrane dane oraz obiekt \emph{Parameters}, który pozwala sparametryzować zapytanie do serwera. Obiekt ten aktualnie posiada tylko jeden atrybut, który określa ile meczów z przeszłości dla danej drużyny brać pod uwagę przy pobieraniu danych.
        
        Dodatkowo, druga z funkcji pozwala na określenie drużyny, dla której mają być pobrane dane. Oto jak wygląda przykładowe wywołanie obu tych funkcji.
        
        \begin{lstlisting}[language=Python, label={lis:example_usage}, caption=Przykładowe wywołanie funkcji do pobrania danych]
        data_aggregator = DataAggregator()
        
        # pobierz dane dla sezonu 2011 i 2012
        data_aggregator.get_data_for_seasons([Season.y2011, Season.y2012], Parameters(no_last_matches=3))
        
        # pobierz dane dla Arsenalu z sezonu 2011 i 2012
        data_aggregator.get_data_for_team_in_seasons(Team.Arsenal, [Season.y2011, Season.y2012], Parameters(no_last_matches=3))
        \end{lstlisting}
        
        \noindent Zapytanie do serwera pozwala na pobranie danych jednorazowo dla konkretnego sezonu lub drużyny. Natomiast jak można zauważyć na listingu \ref{lis:example_usage}, klient ma możliwość wykonania zapytania dla kilku sezonów naraz. Tę możliwość udostępnia klasa \emph{DataAggregator}, która wewnętrznie wykonuje osobno zapytanie dla każdego sezonu i agreguje otrzymane dane w jeden zbiór, który zwracany jest klientowi.
        
        W celu pobrania danych, klasa \emph{DataAggregator} korzysta z klasy znajdujacęj się we wcześniej wspomnianym pliku \texttt{data\_fetcher.py} o nazwie \emph{DataFetcher}. Na listingu \ref{lis:data_fetcher} widać fragment tej klasy. Przy wykonywaniu zapytania, najpierw korzysta ona z funkcji pomocniczej do stworzenia odpowiedniego url'a, a później wykonuje faktyczne zapytanie i zwraca jego wynik. Warto również zwrócić uwagę na mechanizm z pamięcią podręczną. Po otrzymaniu odpowiedzi od serwera, jest ona zapisywana w pamięci pod odpowiednim kluczem. Przed wykonaniem zapytania do serwera, sprawdzane jest, czy nie zostało już wcześniej wykonane zapytanie z takimi samymi parametrami. Jeżeli tak, zwracany jest jego wynik.
        
        
        \begin{lstlisting}[language=Python, label={lis:data_fetcher}, caption=Fragment klasy \emph{DataFetcher}]
        class DataFetcher:

            def __init__(self):
                self.cache = {}

            def fetch_data_for_season(self, season, params):
                url = get_url_for_matches_in_season(season.value, params)
                if url in self.cache:
                    return self.cache[url]
                data = requests.get(url).text
                self.cache[url] = data
                return data
        \end{lstlisting}
    
        
        
        \subsection{Tworzenie zbioru cech (TBC)}
        \noindent Zbiór cech jest jednym z kluczowych czynników, które wpływają na jakość algorytmów uczenia maszynowego. Odpowiedni dobór oraz selekcja i segregacja to droga do uzyskania dobrej predykcji. Jednak wybór odpowiednich cech nie jest łatwym zadaniem i zazwyczaj zajmuje on dużo czasu i zasobów. Także w naszym problemie, dobór cech był starannie dokonany. Piłka nożna do bardzo rozbudowana gra i z jednej partii między drużynami można wyciągnąć nieskończoną ilość danych, poprzez najbardziej oczywiste jak liczba strzałów, po te mniej jak ilość minut spędzonych na swojej połowie przez danego gracza lub średni wiek piłkarza w danej drużynie. Z racji, że posiadaliśmy dość rozbudowaną bazę pochodzącą z różnych źródeł, ostatecznie wybraliśmy 30 cech, które odpowiednio zagregowaliśmy a następnie przekazaliśmy na wejście naszych algorytmów uczenia maszynowego. Lista tych cech wraz z ich krótkim wyjaśnieniem:
        
        \begin{itemize}
            \item \emph{avg\_away\_win\_odds}, \emph{avg\_home\_win\_odds}, \emph{avg\_draw\_odds}: średnia wartość kursów oferowanych na zwycięstwo danej drużyny, remis oraz wygraną drugiej drużyny,
            \item \emph{home\_elo\_rating}, \emph{away\_elo\_rating}: \emph{EloRating} dla obu drużyn,
            \item \emph{home\_players\_avg\_age}, \emph{away\_players\_avg\_age}: średni wiek graczy w drużynach, 
            \item \emph{home\_players\_avg\_rating}, \emph{away\_players\_avg\_rating}: średnia siła zawodników danej drużyny (statystyki z gry Fifa), 
            \item \emph{home\_team\_score}, \emph{away\_team\_score}: siła drużyn (statystyki z gry Fifa, takie jak: szybkość budowania ataku, ilość wymienianych podań...), 
            \item \emph{home\_avg\_corners}, \emph{away\_avg\_corners}: średnia liczba rzutów rożnych na mecz danej drużyny w ciągu ostatnich X meczów, 
            \item \emph{home\_avg\_shots}, \emph{away\_avg\_shots}: średnia liczba oddanych strzałów na mecz danej drużyny w ciągu ostatnich X meczów, 
            \item \emph{home\_won\_games}, \emph{away\_won\_games}: liczba wygranych przez drużynę spotkań z ostatnich X meczów, 
            \item \emph{home\_tied\_games}, \emph{away\_tied\_games}: liczba remisów w ostatnich X meczach, 
            \item \emph{home\_lost\_games}, \emph{away\_lost\_games}: liczba przegranych w ostatnich X meczach, 
            \item \emph{home\_scored\_goals}, \emph{away\_scored\_goals}: liczba strzelonych przez drużynę goli w ostatnich X meczach, 
            \item \emph{home\_team\_last\_season\_points}, \emph{away\_team\_last\_season\_points}: zdobyte punkty przez drużynę w ostatnim sezonie, 
            \item \emph{home\_team\_seasons\_played}, \emph{away\_team\_seasons\_played}: liczba sezonów, które dana drużyna gra w Premier League, 
            \item \emph{home\_direct\_wins}: liczba zwycięstw drużyny gospodarza nad drużyną gościa w ostatnich X meczach rozgrywanych przeciwko sobie,
            \item \emph{away\_direct\_wins}: liczba zwycięstw drużyny gościa nad drużyną gospodarza w ostatnich X meczach rozgrywanych przeciwko sobie, 
            \item \emph{direct\_draws}: liczba remisów drużyny gościa z drużyną gospodarzy w ostatnich X meczach rozgrywanych przeciwko sobie. 
        \end{itemize}   
        
        \noindent Parametr X jest parametrem, który można określić przy inicjowaniu ładowania danych przy pomocy obiektu \emph{Parameters}.
        
        \subsection{Testowanie jednostkowe}
        \noindent W module tym testowane są cztery kluczowe pliki przy pomocy testów jednostkowych. Do tego celu została wykorzystana biblioteka \emph{unittest}. Odpowiadające tym plikom testy znajdują się w folderze \texttt{test/} i prezentują się następująco:
        
        \begin{itemize}
            \item \texttt{test\_api.py} - testuje plik \texttt{api.py} (TBC).
            \item \texttt{test\_data\_aggregator.py} - testuje plik \texttt{data\_aggregator.py} (TBC).
            \item \texttt{test\_data\_fetcher.py} - testuje plik \texttt{data\_fetcher.py} (TBC).
            \item \texttt{test\_preprocessor.py} - testuje plik \texttt{preprocessor.py} (TBC).
        \end{itemize}
        
        Testy jednostkowe dają nam pewność, że otrzymywane z tego modułu dane są poprawne - nie zawierają żadnych błędów i braków. Jest to kluczowe, aby zagwarantować poprawność implementacji modeli, które z tych danych będą korzystać.
        
    \section{Algorytmy}
    Po przetworzeniu danych i ich przygotowaniu kolejnym krokiem w realizacji projektu jest dostosowanie wydajnych algorytmów. W tej sekcji zaprezentowane zostaną najlepsze próby rozwiązania dla postawionego zadania. Przedstawione zostaną kolejno cztery algorytmy, które zostały wypróbowane i zwróciły rzetelne rezultaty. Zostanie zaprezentowana ich struktura oraz parametry dobrane w taki sposób by maksymalizować jakość rozwiązania oraz osiągane wyniki.
        \subsection{Sztuczna Sieć Neuronowa - SNN}
        \label{SNN-param}
        Pierwszym podejściem, które zostanie zaprezentowane jest zbudowana na potrzeby zadania sztuczna sieć neuronowa. Przechodząc do budowy sieci, można ją opisać jako jednokierunkowa, sekwencyjna sieć posiadająca dwie warstwy ukryte, warstwę wejściową i wyjściową. Po każdej warstwie poza wyjściową, użyta jest warstwa normalizacji wsadowej (\english{BatchNormalization}) \cite{BatchNormalization}. Dodatkowo po pierwszej (i tylko po niej) zastosowano technikę Monte Carlo wraz z losowym porzucaniem połączeń pomiędzy neuronami (\english{Monte Carlo (MC) Dropout}) \cite{MCDropout} \cite{Dropout} \cite{Dropout2}. W pierwszej warstwie, warstwie wejściowej zastosowano 40 neuronów, które przepuszczają swoją kombinację danych wejściowych przez funkcję aktywacji \definicja{relu}: \[ReLU(x) = max(0, x)\] co sprawia, że funkcja na wyjściu przekazuje wartości nieujemne. Kolejno w warstwie \definicja{MCDropout} ustawiono współczynnik porzucania równy 0.3. Następna warstwa ukryta składała się z 40 neuronów i funkcji aktywacji \definicja{selu} \cite{SELU} a jej formuła ma się następująco:
        \[
        SELU(x) = \lambda
        \begin{cases}
            x &  \text{if}\ x > 0\\
            \alpha e^{x} - \alpha &  \text{if}\ x \le 0
        \end{cases}
        \]
        gdzie:
        \begin{center}
            $\alpha \approx 1.6732632423543772848170429916717$ \\ 
            $\lambda \approx 1.0507009873554804934193349852946$
        \end{center}
        Następnie w warstwie ukrytej również zastosowano funkcję aktywacji \definicja{selu}, lecz tym razem umieszczono w niej zaledwie 5 neuronów. Warstwa wyjściowa składała się z ilości neuronów odpowiadającej ilości klas równej 3 (Draw, HomeWin, AwayWin), a funkcja aktywacji to funkcja \definicja{softmax}:
        \[
        \hat{p}_{k} = \sigma(s(x))_{k} = \frac{exp \big(s_{k}(x)\big)}{\sum_{j=1}^{K}exp \big(s_{j}(x)\big)}
        \]
        gdzie:
        \begin{itemize}
            \item K to liczba klas,
            \item s(x) to wektor zawierający wyniki każdej klasy dla instancji x,
            \item $\sigma(s(x))_{k}$ jest szacowanym prawdopodobieństwem, że instancja x należy do klasy k, biorąc pod uwagę wyniki każdej klasy dla tej instancji.
        \end{itemize}
        Model sekwencyjny nauczono przy pomocy optymalizatora \definicja{nadam} \cite{adam} \cite{nadam} wraz z funkcją straty \definicja{sparse categorical corossentropy}. Wstępnie ustawione zostało 80 epok, które miały za zadanie uzyskać najlepszy wynik dla naszego problemu, jednak wczesne zatrzymywanie (\english{early stopping}) pozwoliło na zatrzymanie treningu już na dwudziestej pierwszej epoce chroniąc model przed przeuczeniem (\english{overfitting}).
        
        Po treningu, w celu testowania i predykcji wyników, zastosowana warstwa \definicja{MCDropout} pozwala na kontynuowanie porzucania nawet w fazie potreningowej (w przeciwieństwie do podstawowej techniki \definicja{Dropout}, która po nauczeniu sieci, w fazie testowania nie porzucała neuronów - nie spełniała już żadnej funkcji) i dzięki temu zastosowano technikę, w której nowy przykład, którego klasę chce się przewidzieć, jest przepuszczany przez sieć 100 razy, każdy wynik z poszczególnego przebiegu jest przechowywany, a następnie uśredniany dla każdej z możliwych klas. Po tej operacji zachowane są bardziej rzetelne wyniki, które przełożyły się na lepsze rezultaty na zbiorze testowym.
        
        Schemat sieci można przedstawić w postaci tabeli \ref{tab:SNNTable}
        \begin{table}[H]
            \centering
            \caption{Schemat SNN}
            \label{tab:SNNTable}
            \begin{tabular}{|c|c|c|}
            \hline
                Layer (type) &  Output Shape & Param \#\\ \hline \hline
                dense (Dense) & (None, 40) & 1240 \\ \hline
                batch\_normalization (BatchNormalization) & (None, 40) & 160 \\ \hline 
                mc\_dropout (MCDropout) & (None, 40) & 0 \\ \hline         
                dense\_1 (Dense) & (None, 40) & 1640 \\ \hline      
                batch\_normalization\_1 (BatchNormalization) & (None, 40) & 160 \\ \hline
                dense\_2 (Dense) & (None, 5) &  205 \\ \hline       
                batch\_normalization\_2 (BatchNormalization)  & (None, 5) &  20 \\ \hline
                dense\_3 (Dense) & (None, 3) &  18 \\ \hline \hline 
            \end{tabular}
            	\begin{tabular} {| c |}
                Total params: 3,443 \\
                Trainable params: 3,273 \\
                Non-trainable params: 170 \\
                \hline
                \end{tabular}
        \end{table}
        \subsection{SVM}
        Kolejne podejście, które brano pod uwagę i testowano, był algorytm \definicja{SVM}. W podejściu tym skupiono się na znalezieniu trzech najlepszych parametrów (C, $\gamma$, jądro (\english{kernel})). W celu znalezienia tych wartości, zastosowano technikę losowego przeszukiwania siatki (\english{Randomized Search CV}) \cite{SKcv}, której jako punkt odniesienia zaaplikowano metrykę \definicja{f1\_macro} \cite{SKf1}. Zbiór walidacyjny potrzebny do szacowania wyników oraz porównywania dobranych parametrów podczas szukania, został przedstawiony w sekcji \ref{section:ocenaWynikow} i dotyczył on dzielenia zbioru danych na następujące po sobie bloki.
        
        Po fazie przeszukiwania, wybrane zostały najlepsze parametry, które przedstawiają się następująco:
        \begin{table}[H]
            \centering
             \caption{Parametry SVM}
            \label{tab:my_label}
            \begin{tabular}{| c c |}
            \hline
                 Parametr & wartość \\ \hline \hline
                 C & 3.560291686892903 \\ \hline
                 $\gamma$ & 0.0030492848805430566 \\ \hline
                 jądro & rbf \\ \hline
            \end{tabular}
        \end{table}
        \subsection{Alg3}
        \subsection{Alg4}
\chapter{Wyniki oceny eksperymentalnej}

\noindent W tej sekcji zostaną przedstawione oceny predykcji wyników meczy piłkarskich osiągnięte na danych testowych przez wybrane algorytmy uczące się wchodzące w skład systemu. 

Chcemy przypomnieć, że sport, jakim jest piłka nożna, należy do dyscyplin bardzo złożonych, charakteryzujących się ogromną liczbą zmiennych, które są słabo przewidywalne i trudne do uwzględnienia. Ponadto w rzeczywistości wynik predykcji meczu  determinowany jest pewną dozą szczęścia dla jednej z drużyn. Inne czynniki są również trudne do uwzględnienia. Warunki pogodowe, dyspozycja danego zawodnika oraz nastawienie każdego z graczy jest często czynnikiem kluczowym, lecz niestety niemożliwym do uchwycenia w przypadku typowych danych historycznych, a tym bardziej w celu wykorzystania do predykcji wyniku spotkania. Jednak istnieją również czynniki reprezentowane przez atrybuty, które mają realny wpływ na wynik i właśnie je postarano się w tej pracy zidentyfikować, uwzględnić oraz na ich podstawie dokonywać obliczeń i zwracać rezultaty o potencjalnym zwycięzcy. Dane przygotowane i przetworzone (patrz opis w poprzednich rozdziałach) były podstawą do osiągnięcia wyników naszego systemu, które przedstawiono w poszczególnych podsekcjach dla wybranych algorytmów. 

W posiadanych danych występują 3 następujące klasy, które trzeba przewidzieć: 
\begin{itemize}
    \item \english{Draw} - oznacza remis pomiędzy drużynami (etykieta 0)
    \item \english{HomeWin} - oznacza, że drużyna definiowana jako gospodarz odniosła zwycięstwo (etykieta 1)
    \item \english{AwayWin} - oznacza, że drużyna definiowana jako gość odniosła zwycięstwo (etykieta 2)
\end{itemize}

Liczba przykładów z poszczególnych klas w zbiorze danych przed operacją odlosowania przedstawiono w tabeli \ref{tab:przedOdlosowaniem} 

\begin{table}[H]
    \centering
    \caption{Liczba przykładów  w poszczególnych klasach przed odlosowaniem}
    \label{tab:przedOdlosowaniem}
    \begin{tabular}{| c  c |}
    \hline
         Draw & 586 \\
         HomeWin & 1021 \\
         AwayWin & 662 \\\hline
    \end{tabular}
\end{table}
Jak widać, klasa oznaczająca wygraną drużyny gospodarzy, posiada prawie dwukrotnie więcej przykładów niż pozostałe klasy. Dlatego powyższy zbiór danych został zmniejszony w taki sposób, że najliczniejsza klasa została w przybliżeniu wyrównana do klasy mniej licznej uzyskując w ten sposób bardziej zbalansowane dane, w których nie ma jednej, bardzo dominującej klasy. Metoda wykorzystana do zmniejszenia danych to losowe usunięcie elementów z klasy najbardziej licznej (odpowiednik metod z grupy \english{undersampling}) i właśnie taki zbiór został wykorzystany w dalszym etapie algorytmów (wyniki operacji można zaobserwować w tabeli \ref{tab:poOdlosowaniu}). 

\begin{table}[H]
    \centering
    \caption{Liczba przykładów w poszczególnych klasach po odlosowaniu}
    \label{tab:poOdlosowaniu}
    \begin{tabular}{| c  c |}
    \hline
         Draw & 586 \\
         HomeWin & 662 \\
         AwayWin & 662 \\\hline
    \end{tabular}
\end{table}
Kolejnym krokiem było przygotowanie odpowiednich zbiorów (testowy, walidacyjny oraz treningowy) i rozkład poszczególnych klas w tych zbiorach można zobaczyć w tabeli \ref{tab:rozkłądDanych}.

\begin{table}[H]
\caption{Rozkład danych w odpowiednich zbiorach}
\label{tab:rozkłądDanych}
\centering
\begin{tabular}{| c | c c c |}
\hline
    Klasa & zbiór treningowy & zbiór walidacyjny & zbiór testowy \\ \hline
   Draw & 489 & 32 & 65 \\
   HomeWin & 572 & 31 & 59  \\
   AwayWin & 562 & 33 & 67 \\ \hline
\end{tabular}
\end{table}


We wszystkich algorytmach próbowano również sposobu na nadlosowanie przykładów uczących metodą GlobalCS \cite{GlobalCS} i niestety w każdym z algorytmów rezultaty na wartości trafności (\english{accuracy}) były niższe, dlatego zrezygnowano z tej techniki.

\section{Sztuczne sieci neuronowe}
\label{SNN-results}
\noindent W tej podsekcji, przedstawione zostaną wyniki, które udało się osiągnąć dla struktury sztucznej sieci neuronowej, której parametry wraz z opisem zostały przedstawione w sekcji \ref{SNN-opis} oraz \ref{SNN-param}. Ogólna dokładność - trafność predykcji (\english{accuracy}), obliczona na zbiorze testowym, stanowiącym 10\% z dostępnego zbioru danych wyniosła \definicja{50.79\%}, co w ogólności jest wynikiem satysfakcjonującym, ponieważ w chwili w której osoba zainteresowana predykcją zgadywała by wynik spotkania z równym prawdopodobieństwem wystąpienia jednego z trzech rezultatów, średnia trafność wyniosłaby około 33\%, tak więc wartość powyżej tej liczby jest czymś więcej niż opcją zwykłego zgadywania wyniku. Ponadto inne algorytmy nie doprowadzały do dokładności radykalnie wyższych. 

Wartości innych miar przedstawiono w tabeli \ref{tab:SNNscore}.

\begin{table}[H]
    \centering
    \caption{Wyliczone średnie wartości miar klasyfikacyjnych dla SNN}
    \label{tab:SNNscore}
    \begin{tabular}{| c | c |}
    \hline
         Precyzja (\english{precision}) &  50.41\%\\
         \hline
         Czułość (\english{recall}) &  51.06\%\\
         \hline
         Wartość Fscore &  49.78\%\\
         \hline
    \end{tabular}
\end{table}


Przypomnijmy że miarę precyzji można interpretować jako stosunek $\frac{tp}{tp + fp}$, gdzie tp to liczba poprawnie sklasyfikowanych przykładów, a fp to liczba niepoprawnie sklasyfikowanych przykładów negatywnych jako klasy pozytywnej. Czułość z kolei można interpretować jako stosunek $\frac{tp}{tp + fn}$, gdzie tp jest liczbą poprawnie sklasyfikowanych przykładów pozytywnych, a fn liczba niepoprawnie sklasyfikowanych przykładów pozytywnych jako predykcji w klasie negatywnej -- czyli jest to lokalna dokładność rozpoznania klasy. Fscore interpretuje się jako ważoną średnią harmoniczną precyzji i czułości. Dodatkowo, warto zaznaczyć, że miary te są obliczane na podstawie nieważonej średniej poszczególnych wyników z tych miar i właśnie dlatego, tabela \ref{tab:SNNscore} zawiera pojedyncze wartości takich średnich po klasach. 
Macierz pomyłek przedstawia się następująco:

\begin{center}
\begin{table}[H]
\renewcommand{\arraystretch}{1.5}
\caption{Macierz pomyłek dla SNN}
\label{tab:macierzSNN}
\begin{center}
\begin{tabular}{|c|c|c|c|c|}
   \cline{3-5} 
   \multicolumn{1}{c}{} & & \multicolumn{3}{c|}{Predicted} \\ \cline{3-5}
   \multicolumn{1}{c}{} & & Draw & HomeWin & AwayWin \\ \hline
   
   {Observed/actual}
   & Draw & 20 & 23 & 22 \\ \cline{2-5}
   & HomeWin & 12 & 37 & 10  \\ \cline{2-5}
   & AwayWin & 10 & 17 & 40 \\ \hline
\end{tabular}
\end{center}
\end{table}
\end{center}


Można zauważyć, że najwięcej błędów popełnianych w predykcji, jest w momencie, kiedy faktyczna klasa symbolizuje remis pomiędzy drużynami. 

Po procesie nauki sieci postanowiono dodatkowo sprawdzić wpływ danych cech wejściowych na predykcję danego wyniku i odkryć, które atrybuty odgrywały kluczowe role dla predykcji konkretnego wyniku. Dokonaliśmy tego przy użyciu wartości Shapleya \cite{shapley} oraz biblioteki w języku Python \definicja{Shap}.
\begin{figure}[H] 
        \centering\includegraphics[width=10cm,height=6cm]{figures/ShapSNN.png}
        \caption{Wartości Shapleya dla SNN}\label{Shap-SNN}
\end{figure}

Jak można zauważyć na rysunku \ref{Shap-SNN}, atrybut \textit{home\_elo\_rating} oraz \textit{away\_elo\_rating} miały największe znaczenie (wartość Shapleya), które w największym stopniu wpływa na wyniki predykcji sieci. Nie jest to zaskakujący rezultat, ponieważ atrybut ten jest odzwierciedleniem ogólnej siły drużyny w danym meczu oraz aktualizowany jest co spotkanie więc można było się spodziewać dużego znaczenia tej cechy. Ponadto jego przydatność wskazywano w przeglądanej literaturze o analizie rozgrywek piłkarskich. Dodatkowo, warto zauważyć, że cecha określająca liczbę punktów w zeszłych sezonach danej drużyny również była uznana za mocno wpływającą na predykcje (choć  mniej niż wartości atrybutu elo\_rating) i można ją zinterpretować jako dotychczasowy sposób radzenia sobie danej drużyny w analizowanej lidze.

\section{Metoda wektorów wspierających}
\noindent Ta sekcja będzie prezentować wyniki dla algorytmu SVM. Dokładność (\english{accuracy}) na  zbiorze testowym (takim samym jak poprzednio) wyniosła 49.74\%. Dodatkowe miary, które interpretuje się również tak samo jak w podsekcji \ref{SNN-results}, wyglądają następująco:

\begin{table}[H]
    \centering
    \caption{Wyliczone średnie wartości miar dla SVM}
    \label{tab:SVMscore}
    \begin{tabular}{| c | c |}
    \hline
         Precyzja (\english{precision}) &  48.79\%\\
         \hline
         Czułość (\english{recall}) &  49.90\%\\
         \hline
         Wartość Fscore &  48.45\%\\
         \hline
    \end{tabular}
\end{table}
Dodatkowo, macierz pomyłek została przedstawiona w tabeli \ref{tab:macierzSVM}
\begin{center}
\begin{table}[H]
\renewcommand{\arraystretch}{1.5}
\caption{Macierz pomyłek dla SVM}
\label{tab:macierzSVM}
\begin{center}
\begin{tabular}{|c|c|c|c|c|}
   \cline{3-5} 
   \multicolumn{1}{c}{} & & \multicolumn{3}{c|}{Predicted} \\ \cline{3-5}
   \multicolumn{1}{c}{} & & Draw & HomeWin & AwayWin \\ \hline
   
   {Observed/actual}
   & Draw & 18 & 21 & 26 \\ \cline{2-5}
   & HomeWin & 13 & 35 & 11  \\ \cline{2-5}
   & AwayWin & 10 & 15 & 42 \\ \hline
\end{tabular}
\end{center}
\end{table}
\end{center}
Również przy użycia tego klasyfikatora, prawidłowa predykcja remisu jest najsłabsza. Predykcja zwycięstw oraz porażek jest dużo bardziej skuteczna.

Również dalej wykorzystano wartości Shapleya \cite{shapley} z użyciem  biblioteki w języku Python \definicja{Shap}  w celu określenia, które kombinacje cech miały największy wpływ na otrzymywany wynik w naszym algorytmie.

\begin{figure}[H] 
        \centering\includegraphics[width=10cm,height=6cm]{figures/ShapSVM.png}
        \caption{Wartości Shapleya dla SVM}\label{Shap-SVM}
\end{figure}
W tym przypadku, również wartości elo\_rating odgrywały kluczową rolę dla algorytmu SVM. Interpretacja może być podobna, gdyż wartość ta to zagregowana i wyliczona wartości siły i zdolności danej drużyny, czyli mająca realny wpływ na to, jak dana drużyna ma aktualnie predyspozycje oraz zdolności. Ponadto wysoką wartość przyjęły atrybuty takie jak liczba punktów danych drużyn w poprzednim sezonie oraz liczba punktów danej drużyny w aktualnie rozgrywanym sezonie. 

%Wszystkie te czynniki, oraz dobór parametrów dały rezultaty jak przedstawiono powyżej.

\section{Regresja logistyczna}
\noindent W tej sekcji przedstawione zostaną wyniki osiągnięte przez algorytm regresji logistycznej z parametrami modelu opisanymi szczegółowo w sekcji \ref{tab:params_lr}.

Testy dla algorytmu regresji logistycznej zostały przeprowadzone na kilku wersji zbiorów danych. Oprócz  oryginalnych zbiorów, rozważono pomniejszone zbiory (jak w poprzednich testach), w których pomniejszenie polegało na losowym usunięciu przykładów z klasy najbardziej licznej (\english{undersampling}). Spróbowano także wykorzystać nadlosowanie przykładów uczących metodą GlobalCS w ten sposób, że liczba obserwacji z klas najmniej licznych została wyrównana do liczby obserwacji z klasy najbardziej licznej (\english{oversampling}). Dla każdej z tej metod nauczono od podstaw algorytm klasyfikujący oraz przeprowadzono predykcję na zbiorze testowym (liczność zbioru testowego wyniosła 10\% obserwacji całego zbioru danych).

\begin{table}[H]
    \centering
    \caption{Wyliczone średnie wartości miar w różnych podejściach dla danych niezbalansowanych dla algorytmu regresji logistycznej}
    \label{tab:LRSampling}
    \begin{tabular}{| c | c | c | c | c |}
    \hline
        Podejście & Dokładność & Prezycja & Czułość & Wartość Fscore \\ \hline 
        \hline
        Oryginalny zbiór danych & 46.7\% & 44.06\% & 45.32\% & 43.95\% \\
        \hline
        \textit{Undersampling} & 52.88\% & 52.21\% & 53.16\% & 52.3\% \\
        \hline
        \textit{GlobalCS} & 45.81\% & 42.75\% & 42.74\% & 42.18\% \\
         \hline
    \end{tabular}
\end{table}

Analizując wyniki otrzymane poprzez wykorzystanie różnych podejść do problemu niezbalansowanych danych można stwierdzić, że dla algorytmu regresji logistycznej zdecydowanie najlepiej sprawdza się metoda polegająca na usunięciu obserwacji z klas bardziej licznych, która jest wykorzystana także w innych podrozdziałach.

Ponadto wykonany został kolejny test sprawdzający ile meczów wstecz dla danej drużyny powinniśmy brać pod uwagę przy wyliczaniu cech dla naszego algorytmu, tak aby maksymalizować średnie wartości dokładności, precyzji oraz czułości.

\begin{table}[H]
    \centering
    \caption{Znalezienie odpowiedniej liczby meczów dla wyliczanych cech w algorytmie regresji logistycznej}
    \begin{tabular}{| c | c | c | c | c |}
    \hline
        Liczba meczów wstecz & Dokładność & Prezycja & Czułość & Wartość Fscore \\ \hline 
        \hline
        Trzy mecze wstecz & 52.88\% & 52.21\% & 53.16\% & 52.3\% \\
        \hline
        Cztery mecze wstecz & 51.31\% & 50.97\% & 51.43\% & 50.98\% \\
        \hline
        Pięć meczów wstecz & 52.36\% & 51.71\% & 52.53\% & 51.77\% \\
         \hline
    \end{tabular}
\end{table}

Przyglądając się wynikom osiągniętym w powyższej tabeli dla dalszych testów algorytmu wykorzystany został zbiór danych, dla którego odpowiednie statystyki wyliczone są na podstawie trzech meczów wstecz.

Dokonano także przeglądu cech wykorzystywanych w algorytmie, po której to analizie zdecydowano się usunąć ze zbioru treningowego jak i uczącego cechy: \textit{home\_direct\_wins}, \textit{away\_direct\_wins}, \textit{direct\_draws}, gdyż cechy te nie dostarczały istotnych informacji algorytmowi oraz wpływały negatywnie na osiągane przez niego rezultaty. Po powyższej redukcji dokładność klasyfikacji na zbiorze testowym wyniosła 53.40\%. Poszczególne średnie wartości miary precyzji, czułości oraz wartości Fscore prezentują się następująco:

\begin{table}[H]
    \centering
    \caption{Wyliczone średnie wartości miar dla algorytmu regresji logistycznej}
    \label{tab:LRscore}
    \begin{tabular}{| c | c |}
    \hline
         Precyzja (\english{precision}) &  52.81\%\\
         \hline
         Czułość (\english{recall}) &  53.80\%\\
         \hline
         Wartość Fscore &  52.79\%\\
         \hline
    \end{tabular}
\end{table}

\newpage

Macierz pomyłek dla tego algorytmu prezentuje się następująco:

\begin{center}
\begin{table}[H]
\renewcommand{\arraystretch}{1.5}
\caption{Macierz pomyłek dla algorytmu regresji logistycznej}
\begin{center}
\begin{tabular}{|c|c|c|c|c|}
   \cline{3-5} 
   \multicolumn{1}{c}{} & & \multicolumn{3}{c|}{Predicted} \\ \cline{3-5}
   \multicolumn{1}{c}{} & & Draw & HomeWin & AwayWin \\ \hline
   
   {Observed/actual}
   & Draw & 23 & 18 & 24 \\ \cline{2-5}
   & HomeWin & 10 & 40 & 9  \\ \cline{2-5}
   & AwayWin & 15 & 13 & 39 \\ \hline
\end{tabular}
\end{center}
\end{table}
\end{center}

Na podstawie powyższych wyników można wyciągnąć podobne wnioski jak we dwóch wcześniejszych podejściach. Algorytm zdecydowanie radził sobie najgorzej z typowaniem remisów w przeciwieństwie do typowania zwycięstwa którejś z drużyn. \\

Ważność odpowiednich cech dla algorytmu została ustalona poprzez wykorzystanie metody \textit{coef\_} dla modelu regresji logistycznej z biblioteki \textit{scikit-learn}. Wartości z poniższej tabeli należy odpowiednio interpretować: im wyższa wartość współczynnika dla danej cechy tym ta cecha odgrywała większą rolę w procesie predykcji ostatecznej klasy dla danej obserwacji. Analogicznie im mniejsza wartość współczynnika tym ważność danej cechy w całym procesie była mniejsza. Tabela zbierająca ważności cech prezentuje się następująco:

\begin{table}[H]
        \caption{Ważność cech modelu regresji logistycznej}
        \centering
        \begin{tabular}{c c c}
        \toprule
            Cecha & Współczynnik \\
        \midrule
            avg\_away\_win\_odds & 0.0232 \\
            home\_avg\_shots & 0.0215 \\
            avg\_home\_win\_odds & 0.0203 \\
            home\_scored\_goals & 0.0198 \\
            away\_avg\_shots & 0.0166 \\
            away\_players\_avg\_age & 0.016 \\
            home\_players\_avg\_rating & 0.0159 \\
            away\_players\_avg\_rating & 0.0148 \\
            home\_team\_seasons\_played & 0.0147 \\
            away\_won\_games & 0.0108 \\
            away\_avg\_corners & 0.0105 \\
            avg\_draw\_odds & 0.0097 \\
            home\_avg\_corners & 0.0077 \\
            away\_lost\_games & 0.0073 \\
            away\_tied\_games & 0.0064 \\
            away\_scored\_goals & 0.0063 \\
            home\_tied\_games & 0.0039 \\
            away\_team\_score & 0.0039 \\
            home\_players\_avg\_age & 0.0038 \\
            away\_team\_seasons\_played & 0.0034 \\
            home\_team\_score & 0.0033 \\
            away\_team\_last\_season\_points & 0.0021 \\
            home\_lost\_games & 0.0019 \\
            home\_elo\_rating & 0.0019 \\
            away\_elo\_rating & 0.0014 \\
            home\_team\_last\_season\_points & 0.0009 \\
            home\_won\_games & 0.0008 \\
        \bottomrule
        \end{tabular}
        \end{table}

Na podstawie powyższej tabeli można zauważyć, że najwyższy wpływ na dokonaną predykcję mają wartości kursów oferowanych przez zakłady bukmacherskie. Zdawano sobie sprawę, że te atrybuty mogą mieć znaczenie w problemie predykcji wyniku meczu, dlatego nie jest to rezultat zaskakujący. Warto także podkreślić wartości dla cech związanych z liczbą strzałów oraz liczbą zdobytych goli w ostatnich meczach. W przeciwieństwie do poprzednich modeli cechy związane z elo\_ratingiem są nisko klasyfikowane, praktycznie znajdują się na końcu tej listy, co może być pewnym zaskoczeniem. Na podstawie wartości otrzymanych dla cech powiązanych z elo\_ratingiem można wyciągnąć wniosek, że algorytm regresji uznał te cechy za mniej ważne w procesie predykcji na rzecz cech związanych z kursami bukmacherskimi, liczbą strzałów czy ogólną średnią drużyny wyliczoną na podstawie statystyk z gry FIFA.

\section{Las losowy}
\label{resultsRF}
\noindent Sekcja przedstawia  wyniki osiągnięte przez algorytm lasu losowego. Postępowanie w procesie treningu modelu odbyło się analogicznie jak przy algorytmie regresji logistycznej. Dla każdego z trzech wcześniej opisanych podejść: uczenie na oryginalnym zbiorze danych, uczenie na pomniejszonym zbiorze danych oraz uczenie na powiększonym zbiorze danych dokonano procesu nauki od podstaw algorytmu klasyfikującego oraz przeprowadzono predykcję na zbiorze testowym (liczność zbioru testowego wyniosła 10\% obserwacji całego zbioru danych), która pozwoliła na ostateczny wybór wykorzystanego podejścia.

\begin{table}[H]
    \centering
    \caption{Wyliczone średnie wartości miar w podejściu dla danych niezbalansowanych dla algorytmu lasu losowego}
    \label{tab:LRSampling}
    \begin{tabular}{| c | c | c | c | c |}
    \hline
        Podejście & Dokładność & Prezycja & Czułość & Wartość Fscore \\ \hline 
        \hline
        Oryginalny zbiór danych & 50.66\% & 52.1\% & 45.54\% & 41.02\% \\
        \hline
        \textit{Undersampling} & 49.21\% & 48.16\% & 49.22\% & 48.23\% \\
        \hline
        \textit{GlobalCS} & 49.78\% & 44.89\% & 45.59\% & 43.47\% \\
         \hline
    \end{tabular}
\end{table}

Jak można zauważyć na podstawie powyższej tabeli wyniki dla trzech różnych podejść okazały się bardziej zbliżone aniżeli to było w przypadku wcześniej omówionych modeli. Do dalszego testowania zdecydowano się wykorzystać podejście opierające się na wykorzystaniu w procesie uczenia oryginalnego zbioru danych, gdyż cechowało ono się najwyższą wartością dokładności oraz precyzji, choć czułość jest slabsza. Wybór ten zostanie także uargumentowany w dalszej części pracy.

Kolejno, podobnie jak we wcześniejszym algorytmie, zdecydowano się zbadać parametr odpowiadający za liczbę meczów wstecz, na podstawie której są wyliczane odpowiednie atrybuty dla przedstawianych algorytmów.

\begin{table}[H]
    \centering
    \caption{Znalezienie odpowiedniej liczby meczów dla wyliczanych cech w algorytmie lasu losowego}
    \begin{tabular}{| c | c | c | c | c |}
    \hline
        Liczba meczów wstecz & Dokładność & Prezycja & Czułość & Wartość Fscore \\ \hline 
        \hline
        Trzy mecze wstecz & 50.66\% & 52.1\% & 45.54\% & 41.02\% \\
        \hline
        Cztery mecze wstecz & 47.58\% & 31.17\% & 41.4\% & 34.78\% \\
        \hline
        Pięć meczów wstecz & 47.14\% & 30.97\% & 41.06\% & 34.53\% \\
         \hline
    \end{tabular}
\end{table}

Na podstawie powyższych wyników można jednoznacznie stwierdzić, że cechy obliczane na podstawie trzech meczów wstecz dla danej drużyny, zdecydowanie najlepiej współpracują z algorytmem lasu losowego. Wartości w każdej kolumnie wierszu pierwszego tabeli są wyższe aniżeli wartości z pozostałych wierszy tabeli. Badanie to ukierunkowuje nas na jednoznaczny wybór tego parametru w dalszej analizie problemu.

Dla algorytmu lasu losowego także zdecydowano przeprowadzić się selekcję cech - próbowano usunąć ze zbioru danych atrybuty odznaczające się najmniejszym wpływem w procesie wyboru ostatecznej klasy dla danej obserwacji, jednak każda taka operacja skutkowała pogorszeniem wartości na przedstawianych metrykach. Ostatecznie zdecydowano się pozostawić wszystkie cechy w zbiorze danych. Algorytm lasu losowego poprzez losowy wybór podzbioru atrybutów w procesie podziału w węźle sam potrafi wyselekcjonować najbardziej pożądane cechy w procesie uczenia, co zostało dokładniej opisane w sekcji \ref{algRandomForest}.

Po przeprowadzonym procesie uczenia i tuningu algorytmu otrzymano następujące wyniki dla przedstawianego algorytmu. Wartość dokładności zmierzona na zbiorze testowym wyniosła 50.66\%. Poszczególne miary precyzji, czułości oraz wartości Fscore przedstawione zostały w tabeli \ref{tab:RFScore}.

\begin{table}[H]
    \centering
    \caption{Wyliczone średnie wartości miar dla algorytmu lasu loswego}
    \label{tab:RFScore}
    \begin{tabular}{| c | c |}
    \hline
         Precyzja (\english{precision}) &  52.1\%\\
         \hline
         Czułość (\english{recall}) &  45.54\%\\
         \hline
         Wartość Fscore &  41.02\%\\
         \hline
    \end{tabular}
\end{table}

Macierz pomyłek dla tego algorytmu prezentuje się następująco:

\begin{center}
\begin{table}[H]
\renewcommand{\arraystretch}{1.5}
\caption{Macierz pomyłek dla algorytmu lasu losowego}
\begin{center}
\begin{tabular}{|c|c|c|c|c|}
   \cline{3-5} 
   \multicolumn{1}{c}{} & & \multicolumn{3}{c|}{Predicted} \\ \cline{3-5}
   \multicolumn{1}{c}{} & & Draw & HomeWin & AwayWin \\ \hline
   
   {Observed/actual}
   & Draw & 4 & 42 & 18 \\ \cline{2-5}
   & HomeWin & 1 & 78 & 19  \\ \cline{2-5}
   & AwayWin & 2 & 30 & 33 \\ \hline
\end{tabular}
\end{center}
\end{table}
\end{center}

To co może się rzucać na pierwszy rzut oka to fakt, że algorytm lasu losowego ma bardzo małą skuteczność w porównaniu do poprzednich algorytmów w rozpoznaniu klasy remisów. Z tego względu zdecydowano się także pokazać proces uczenia na oryginalnym zbiorze danych aniżeli na zbiorze pomniejszonym lub powiększonym o kolejne obserwacje. W tym przypadku doszło do sytuacji opisywanej w podrozdziale \ref{ImbalancedData}. Dla opisywanego algorytmu działaliśmy na danych niezbalansowanych: liczba obserwacji, w których mecz wygrała drużyna grająca na swoim stadionie wyniosła 1021, liczba obserwacji, w których mecz wygrała drużyna grająca na wyjeździe wyniosła 662, liczba obserwacji, w których odnotowano remis w meczu wyniosła 586. W tym przypadku doszło do faworyzowania przez wyuczony klasyfikator klasy dominującej kosztem klasy zdominowanej. Na podstawie macierzy pomyłek możemy zaobserwować, że algorytm przydzielił etykietę ,,HomeWin'' aż 150 obserwacjom ze zbioru 227 przykładów, co stanowi ponad 66\% całego zbioru testowego. W tym przypadku model mimo wysokich wartości dokładności i precyzji nie poradził sobie zbyt dobrze z predykcją etykiet końcowych.


Dla omawianego modelu można jeszcze dokonać spojrzenia na ważność cech modelu, które przyczyniły najbardziej się do predykcji konkretnego wyniku. Do przedstawienia ważności cech wykorzystano metodę \textit{feature\_importances\_} z biblioteki scikit-learn.

\begin{table}[H]
        \caption{Ważność cech modelu lasu losowego}
        \centering
        \begin{tabular}{c c c}
        \toprule
            Cecha & Ważność \\
        \midrule
            avg\_home\_win\_odds & 0.1321 \\
            avg\_away\_win\_odds & 0.1288 \\
            home\_players\_avg\_rating & 0.085 \\
            home\_elo\_rating & 0.0822 \\
            avg\_draw\_odds & 0.0765 \\
            home\_team\_last\_season\_points & 0.0576 \\
            away\_players\_avg\_rating & 0.054 \\
            away\_elo\_rating & 0.0493 \\
            away\_team\_last\_season\_points & 0.033 \\
            away\_players\_avg\_age & 0.0296 \\
            home\_avg\_shots & 0.0266 \\
            home\_avg\_corners & 0.0262 \\
            away\_avg\_shots & 0.0211 \\
            away\_direct\_wins & 0.02 \\
            home\_players\_avg\_age & 0.02 \\
            home\_scored\_goals & 0.0187 \\
            away\_team\_score & 0.0171 \\
            home\_team\_score & 0.0162 \\
            away\_avg\_corners & 0.0151 \\
            away\_scored\_goals & 0.0139 \\
            home\_direct\_wins & 0.0135 \\
            home\_team\_seasons\_played & 0.0112 \\
            away\_team\_seasons\_played & 0.0092 \\
            home\_tied\_games & 0.0075 \\
            direct\_draws & 0.0074 \\
            home\_won\_games & 0.0068 \\
            away\_lost\_games & 0.0064 \\
            away\_won\_games & 0.0059 \\
            home\_lost\_games & 0.0053 \\
            away\_tied\_games & 0.0037 \\
        \bottomrule
        \end{tabular}
\end{table}

Na podstawie powyższej tabeli można zauważyć wysokie znaczenie cech związanych z kursami bukmacherskimi oraz wartością elo\_rating - pokrywa się to z wcześniejszymi wnioskami oraz najważniejszymi cechami, które były przedstawiane przy okazji poprzednich algorytmów. Wysoką ważność ma także zagregowana siła drużyny na podstawie statystyk zawodników z gry FIFA, co może sugerować, że producent dobrze odwzorował istniejącą rzeczywistość w wirtualnym świecie. Również, podobnie jak w algorytmie SNN oraz SVM, dużą ważność przyjęły atrybuty związane z liczbą punktów zdobytych przez daną drużynę w poprzednim sezonie.

\section{Multi-class Rougly Balanced Bagging}
\noindent Sekcja ta stanowi rozszerzenie do sekcji \ref{resultsRF}, w której przedstawiono wyniki modelu lasu losowego na oryginalnym zbiorze danych. Po analizie rezultatów można było dojść do wniosku, że algorytm ten nie radzi sobie zbyt dobrze z problemem danych niezbalansowanych. W celu poprawy tego podejścia zdecydowano przetestować algorytm Multi-class Roughly Balanced Bagging (dalej oznaczany jako MRBB), opisany w sekcji \ref{ImbalancedData}, gdzie jako jako klasyfikator bazowy wykorzystamy algorytm drzewa decyzyjnego.

Po przeprowadzonym procesie uczenia uzyskano następujące wyniki na zbiorze testowym:

\begin{table}[H]
    \centering
    \caption{Porównanie średniej wartości miar dla algorytmu lasu losowego oraz MRBB}
    \begin{tabular}{| c | c | c | c | c |}
    \hline
        Algorytm & Dokładność & Prezycja & Czułość & Wartość Fscore \\ \hline 
        \hline
        Las losowy & 50.66\% & 52.1\% & 45.54\% & 41.02\% \\
        \hline
        MRBB & 49.33\% & 47.84\% & 47.58\% & 47.31\% \\
        \hline
    \end{tabular}
\end{table}

Na podstawie powyższej tabeli możemy zauważyć, że algorytm lasu losowego odznaczał się wyższymi wartościami na kryterium dokładności oraz precyzji oraz niższymi wartościami na kryterium czułości oraz wartości Fscore. Warto spojrzeć także na porównanie macierzy pomyłek dla obu tych podejść przed wyciągnięciem wniosków. Komórka w macierzy pomyłek została zapisana w postaci: wartość komórki z macierzy pomyłek lasu losowego/wartość komórki z macierzy pomyłek algorytmu MRBB.

\begin{center}
\begin{table}[H]
\renewcommand{\arraystretch}{1.5}
\caption{Macierz pomyłek dla algorytmu lasu losowego oraz MRBB}
\begin{center}
\begin{tabular}{|c|c|c|c|c|}
   \cline{3-5} 
   \multicolumn{1}{c}{} & & \multicolumn{3}{c|}{Predicted} \\ \cline{3-5}
   \multicolumn{1}{c}{} & & Draw & HomeWin & AwayWin \\ \hline
   
   {Observed/actual}
   & Draw & 4/21 & 42/20 & 18/23 \\ \cline{2-5}
   & HomeWin & 1/17 & 78/58 & 19/23  \\ \cline{2-5}
   & AwayWin & 2/10 & 30/22 & 33/33 \\ \hline
\end{tabular}
\end{center}
\end{table}
\end{center}

Obserwując wartości odnotowane w macierzy pomyłek zauważamy, że algorytm MRBB zdecydowanie lepiej radzi sobie z przewidywaniem remisów aniżeli algorytm lasu losowego. Również widoczna jest różnica w przydziale etykiety ,,HomeWin'' dla danej obserwacji. Dla tego algorytmu przydzielono jedynie taką etykietę 90 razy, podczas gdy dla modelu lasu losowego taka sytuacja wystąpiła 150 razy. Na podstawie tych obserwacji możemy także wyciągnąć wniosek jak ważna w analizowanym problemie była umiejętność poradzenia sobie z danymi niezbalansowanymi czy to poprzez odsianie bądź dolosowanie obserwacji lub poprzez wykorzystanie rozszerzeń standardowych algorytmów dla danych niezbalansowanych.

\section{Porównanie algorytmów}

\noindent Podsumowując i porównując wszystkie metody - klasyfikatory można stwierdzić, że prowadzą one do w miarę porównywalnych dokładności predykcji (około 49--51\%). Tak jak dyskutowaliśmy, osiągnięcie dużo wyższych trafności okazało się niemożliwe ze względu na trudność samego zadania oraz specyfikę danych. Może to być przedmiotem dalszych badań naukowych. 

Porównywane klasyfikatory różnią się wartościami średnich czułości oraz precyzji oraz rozpoznawaniem poszczególnych klas. To może być przesłanką dla użytkowników -- analityków sportowych, którzy zgodnie z swoimi preferencjami mogą wybrać jeden z klasyfikatorów lub użyć je jako zespół. 

Dodatkowo warto wspomnieć, że podczas testowania algorytmów wykorzystaliśmy technikę dotyczącą przesuwnego zbioru testowego, którego ideę przedstawiono w rozdziale \ref{section:ocenaWynikow}. Technika ta została wypróbowana w sztucznych sieciach neuronowych, jednakże ze względu na niezadowalające rezultaty, nie została ona rozwinięta w odpowiedniej podsekcji. Wyniki dokładności wahały się pomiędzy $37.5\%$ do $51.5\%$ i ostatecznie, klasyfikator ten spisywał się dużo gorzej niż podczas standardowego procesu uczenia na jednokrotnym podziale na zbiór uczący vs. testowy (wtedy rozmiar zbioru uczącego jest większy).
\chapter{Uwagi końcowe}

\noindent W dzisiejszym sporcie coraz większą rolę odgrywa specjalna analiza dokonywana przed każdym spotkaniem w celu lepszego poznania charakterystyki przeciwnika lub swoich szans w starciu z nim. Potrzebne są nowe narzędzia wspomagające analityków dokonujących takich analiz. Mogłyby one znacząco ułatwić zadanie, zwiększając możliwości dobrego wykorzystania odpowiedniej taktyki, poprawiając tym samym szanse na zwycięstwo. 

Celem pracy było utworzenie systemu informatycznego - złożonego oprogramowania, który pozwoliłby określić zwycięzcę konfrontacji między dwoma drużynami z piłkarskiej ligi angielskiej. Wykorzystuje w tym celu historyczne dane, zagregowane z różnych źródeł, na temat drużyn, graczy oraz meczów zebrane i umieszczone w spójnym repozytorium, będącym bazą danych. Modelowi użytkownicy (np. analitycy sportowi) mają otrzymać odpowiednio sparametryzowane algorytmy uczenia maszynowego gotowe do użycie przy pomocy środowiska \textit{Jupyter Notebook}.

Pierwszym etapem budowy powyższego systemu było pobranie wyselekcjonowanych danych do bazy danych, później wybranie odpowiednich cech najbardziej pasujących do zadania, które mogą być wykorzystane w predykcji wyniku spotkania. Następnie dokonano wyboru oraz przeprowadzono uczenie czterech modeli uczenia maszynowego. 

Przeprowadzono ocenę wyników na zbiorze testowym. Z punktu widzenia globalnej trafności najlepiej poradził sobie algorytm regresji logistycznej uzyskując dokładność $53.4\%$. Bardzo dobrze spisała się również zaimplementowana sieć neuronowa, która uzyskała dokładność równą $50.79\%$, oraz algorytm losowego lasu ($50.66\%$). Nieco niższą dokładność predykcji osiągał klasyfikator nauczony algorytm maszyn wektorów nośnych. Mimo wszystko, wymienione algorytmy pozwoliły na osiągnięcie globalnej trafności na dość podobnym poziomie.  W rezultacie,  obserwując różnice w poprawnym przewidywaniu poszczególnych klas, użytkownik może samemu wybrać, który z modeli spełnia jego preferencje lub zaagregować ich predykcje jako zespół klasyfikatorów. Dodatkowo, w przypadku niezbalansowanych danych, najlepszym podejściem okazała się być relatywnie prosta metoda usuwająca przykłady z klasy o największej liczbie przykładów (zauważmy, że dodatkowo sprawdzany algorytm MRBBag wykorzystuje także losowanie ukierunkowane na tzw. redukcję liczności klasy większościowej). 

Po fazie testowania wszystkie algorytmy zostały następnie nauczone na całym dostępnym zbiorze danych oraz dostarczone do powłoki interaktywnej będącej środowiskiem wykonawczym użytkownika, w którym to użytkownik może bez specjalistycznej wiedzy dokonać predykcji nadchodzącego, interesującego go spotkania.

Pomimo że udało się spełnić założenia i cele pracy, nie znaczy to, że system ten nie może być dalej rozwijany. Jednym z usprawnień mogłoby być zintegrowanie większej ilości aktualnych danych, które potencjalnie wpłynęłyby pozytywnie na wyniki predykcji algorytmów. Niezależnie można by zbadać charakterystykę rozkładów danych, aby lepiej poznać źródła trudności dla  uczenia klasyfikatorów.

Innym pomysłem byłoby udoskonalenie tej części oprogramowania, która odpowiedzialna jest za interakcję z użytkownikiem, tak aby nie wymagała ona instalowania i obsługi dodatkowych, zewnętrznych programów. W tej wersji rozwojowej powinno się ułatwić interakcje z mniej doświadczonymi użytkownikami. Ponadto stworzony system można by rozbudować o dodatkowe narzędzia służące do prezentacji statystyk wykorzystanych danych.

Kolejną drogą rozwoju jest rozszerzenie istniejącego zbioru danych o dane pochodzące z innych lig krajowych piłki nożnej. Mimo iż założeniem była predykcja tylko dla ligi angielskiej, nic nie stoi na przeszkodzie, aby umożliwić analizę dla innych drużyn spoza tej stosunkowo małej grupy. Pozwoliłoby to na szersze spojrzenie na mecze piłki nożnej biorąc pod uwagę indywidualny styl każdej z lig.

%--------------------------------------
% Literatura
%--------------------------------------

\bibliographystyle{plain}{\raggedright\sloppy\small\bibliography{bibliografia}}

%--------------------------------------
% Dodatki
%--------------------------------------
\newpage\null\thispagestyle{empty}\newpage
\cleardoublepage\appendix%
\chapter{Repozytorium}
\noindent Repozytorium na stronie \url{https://github.com/Put-EngineeringThesis-soccer-prediction/SoccerMatchPredictor} zawiera oprogramowanie stworzone przez zespół. W folderze \textit{Algorithms} można znaleźć implementacje algorytmów zawartą w plikach \textit{Jupyter Notebook} oraz plik \textit{Interactive.ipynb} zawierający interaktywne środowisko dla użytkowników.

W folderze \textit{DataProjects} znajdują się solucje komponentów  \textit{Match Predict Data Importer} oraz \textit{Match Predict Data Provider} jak i plik zawierający kopie zapasową bazy danych o nazwie \textit{kaggle\_filtered.bacpac}. W folderze \textit{Preprocesssing} znajduję się moduł Pythona z udostępnionymi metodami, które wstępnie przetwarzają dane. Katalog \textit{visualization} zawiera przeprowadzoną wizualizację danych w formacie \textit{Jupyter Notebook} lub wygenerowanym plikiem html.

Folder \textit{Thesis} zawiera pliki źródłowe plików .tex napisanej pracy dyplomowej.
%--------------------------------------
% Informacja o prawach autorskich
%--------------------------------------
\newpage\null\thispagestyle{empty}\newpage

\ppcolophon

\end{document}