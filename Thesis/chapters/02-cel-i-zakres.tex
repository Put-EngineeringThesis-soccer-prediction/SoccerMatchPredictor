\chapter{Cel i zakres pracy}

\noindent Celem pracy jest stworzenie systemu potrafiącego na podstawie danych historycznych z ligi angielskiej określić zwycięzce konfrontacji pomiędzy dwoma drużynami przy pomocy odpowiednio dobranych cech określających każde spotkanie i algorytmów uczenia maszynowego. W cel zrealizowania tego zadania, system musi zarządzać uporządkowanymi i spójnie połączonymi danymi z różnych źródeł na temat meczów piłkarskich, zawodników oraz drużyn z kilku ostatnich sezonów ligi angielskiej.

Modelowym przeznaczeniem systemu jest udostępnienie go analitykom sportowym, stąd jego elementem będzie też środowisko programowe, które zapewni możliwość interakcji z jego odbiorcami. Konkretnie, zakładamy, że po fazie nauczenia algorytmów na danych historycznych, taki analityk będzie mógł wybrać z udostępnionej listy dwie interesujące go drużyny i otrzymać odpowiedź od systemu, kto byłby zwycięzcą takiego pojedynku. Odpowiednio zaprojektowane, interaktywne środowisko ma być na tyle elastyczne, że można zakładać iż odbiorca nie musi posiadać umiejętności programowania. Jednak z uwagi na charakter tego środowiska, powinien umieć się po nim poruszać oraz mieć przynajmniej podstawową wiedzę z analizy danych.


Zakładamy, że dla realizacji omówionych założeń i celu pracy potrzebne będzie stworzenie systemu składającego się z wzajemnie komunikujących się komponentów: 
\begin{itemize}
    \item  Pierwszym z nich jest baza danych, przechowująca dane o drużynach, meczach oraz zawodnikach oraz udostępniająca interfejs pozwalający na pozyskanie z niej interesujących danych. 
    \item Kolejnym komponentem jest moduł do wstępnego przetwarzania danych, którego zadaniem jest korzystanie ze wspomnianego interfejsu oraz na podstawie uzyskanych danych dokonywanie odpowiednich przekształceń, agregacji oraz filtracji w celu stworzenia zbioru cech wykorzystywanego przy algorytmach uczących.
    \item Następnym elementem systemu jest środowisko, które korzysta z przetworzonych wcześniej danych, w którym przygotowywane i testowane są różne algorytmy uczenia maszynowego z wykorzystaniem różnych technik oraz podejść w celu jak najtrafniejszego przewidywania wyniku meczu.
    \item Część zapewniająca interfejs pozwalający na interakcję z modelowym użytkownikiem - analitykiem sportowym.
\end{itemize}

\noindent Sama struktura i budowa pracy jest następująca:
\begin{itemize}
    \item w rozdziale 3 przedstawiono przegląd badań oraz literatury na temat eksploracji danych w piłce nożnej oraz podstawowe pojęcia z tej dziedziny i opisy algorytmów uczenia maszynowego potrzebne do zrozumienia dalszej części pracy,
    \item w rozdziale 4 szczegółowo opisana została architektura oraz narzędzia i technologie wykorzystane przy implementacji wszystkich komponentów składających się na całość systemu,
    \item rozdział 5 składa się z przedstawienia  analiz poszczególnych komponentów systemu, m.in w jaki sposób pobierane i składowane są dane, jak tworzone są cechy oraz jakie algorytmy i w jaki sposób zostały użyte do predykcji wyników,
    \item w rozdziale 6 przedstawione zostały wyniki eksperymentów z testowanie wybranych algorytmów uczenia maszynowego oraz zawarto studium przypadku użycia interaktywnego interfejsu,
    \item rozdział 7 stanowi podsumowanie pracy
\end{itemize}

\section{Wykorzystane narzędzia i technologie}
\begin{itemize}
    \item Języki programowania: \textit{Python}, \textit{C\#}
    \item Biblioteki: \textit{pandas} \cite{reback2020pandas}, \textit{scikit-learn} \cite{scikit-learn}, \textit{tensorflow} \cite{tensorflow2015-whitepaper}, \textit{numpy} \cite{harris2020array}, \textit{multi-imbalance} \cite{MultiImbalance2020}, \textit{shap} \cite{NIPS2017_7062}
    \item Środowiska programistyczne: \textit{PyCharm Community Edition}, \textit{Microsoft Visual Studio Community 2019}, \textit{Jupyter Notebook}, \textit{SQL Server Management Studio}
    \item Bazy danych: \textit{Microsoft SQL Server}, \textit{Azure}
\end{itemize}
\section{Podział pracy i zadań poszczególnych autorów}
\begin{itemize}
    \item \textbf{Arkadiusz Chmura} - dokonanie analizy i wizualizacji danych przechowywanych w bazie oraz stworzenie modułu będącego pośrednikiem pomiędzy komponentem przechowującym dane a środowiskiem testowym dla algorytmów. Moduł ten ma za zadanie udostępniać wygodny interfejs pozwalający na pobieranie danych z konkretnych sezonów, dla wybranych drużyn. Jest on również odpowiedzialny za dobranie i stworzenie cech, które używane są później przy implementacji algorytmów uczenia maszynowego.
    \item \textbf{Bartosz Ciesielski} - wstępne przetworzenie surowych danych z różnych źródeł oraz ich analiza w celu znalezienia różnorodnych braków oraz niespójności. Utworzenie relacyjnej bazy danych i zaimportowanie, przy pomocy programów konsolowych, zestawów danych z różnych repozytoriów w celu uzyskania jednego spójnego zbioru informacji, wykorzystywanego przez algorytmy uczenia maszynowego. Utworzenie aplikacji WebAPI  dostępnej w sieci, z pomocą której umożliwiony jest dostęp do wstępnie agregowanych danych, zwracanych w formacie JSON.
    \item \textbf{Iwo Naglik} - dokonanie analizy algorytmów wykorzystywanych przez literaturowe prace pokrewne oraz wybór tych algorytmów, które dały najlepsze rezultaty w celu dostosowania ich do założeń i specyfiki projektowanego systemu. Stworzenie kodu dzielącego dane uczące na odpowiednie porcje przykładów (zgodnie z etykietami czasowymi) w celu blokowego testowania algorytmów. Zastosowanie różnych technik w celu poradzenia sobie z problemem niezbalansowanych danych. Dostrajanie algorytmów uczących oraz budowa sieci neuronowej wraz z dobraniem najlepszych parametrów. Głębsza selekcja cech i atrybutów na podstawie wyników algorytmów. Końcowa wizualizacja ważności cech dla konkretnego algorytmu przy pomocy wartości Shapleya \cite{shapley}. Zbudowanie interaktywnego notebooka w celu udostępnienia interfejsu graficznego użytkownikowi końcowemu.
    \item \textbf{Bartosz Przybył} - analiza artykułów poruszających tematykę przewidywania meczów rozgrywek sportowych w celu wyselekcjonowania potencjalnych algorytmów uczenia maszynowego dających zastosować się w opisywanym problemie predykcji. Konstrukcja wstępnej listy cech wykorzystywanej w uczeniu algorytmów. Wykorzystanie algorytmów - regresji logistycznej oraz lasu losowego - rozwiązujących problem predykcji wyniku meczu piłkarskiego oraz ich dalsze dostrajanie poprzez selekcję cech i dobór odpowiednich parametrów dla algorytmów. Wyeksportowanie stworzonych modeli gotowych do wykorzystania w interaktywnym notebooku. Udział w testowaniu algorytmów oraz studium przypadku interakcji.
\end{itemize}
W przypadku podziału przy tworzeniu rozdziałów w pracy inżynierskiej, to każda osoba współtworząca projekt opisuje części, które były przez nią implementowane.