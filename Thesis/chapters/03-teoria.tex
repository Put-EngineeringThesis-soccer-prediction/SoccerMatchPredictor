
\chapter{Podstawy teoretyczne}
Rozdział teoretyczny --- przegląd literatury naświetlający stan wiedzy na dany temat. 
\section{Bazy danych i API}
Opisy relacyjnych baz danych i Web API
\section{Wstępne przetwarzanie}
Opis celu wstępnego przetwarzania danych
\section{Sposób oceny wyników}
Metody oceny wyników, które zwracają algorytmy (jakości algorytmów)
\section{Algorytmy uczenia maszynowego}
opisy teoretyczne wykorzystywanych algorytmów


Przegląd literatury naświetlający stan wiedzy na dany temat obejmuje rozdziały pisane na podstawie
literatury, której wykaz zamieszczany jest w części pracy pt.~\emph{Literatura} (lub inaczej \emph{Bibliografia},
\emph{Piśmiennictwo}). W tekście pracy muszą wystąpić odwołania do wszystkich pozycji zamieszczonych w
wykazie literatury. \textbf{Nie należy odnośników do literatury umieszczać w stopce strony.} Student jest
bezwzględnie zobowiązany do wskazywania źródeł pochodzenia informacji przedstawianych w pracy,
dotyczy to również rysunków, tabel, fragmentów kodu źródłowego programów itd. Należy także podać
adresy stron internetowych w przypadku źródeł pochodzących z Internetu.


