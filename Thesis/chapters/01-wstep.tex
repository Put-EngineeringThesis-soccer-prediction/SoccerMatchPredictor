
\chapter{Sztuczna inteligencja w analizie danych o rozgrywkach sportowych}

\noindent Praca podejmuje tematykę eksploracji danych pochodzących z różnych źródeł i występujących w różnych strukturach oraz zastosowań metod uczenia maszynowego w przewidywaniu rezultatu meczów piłkarskich z ligi angielskiej- najczęściej oglądanej w tej dyscyplinie sportu~\cite{ESPN}.

Popularność tej ligi i samej dyscypliny powoduje powstawanie ogromnej ilości danych możliwych do analiz, co czyni ją atrakcyjną dla badaczy. W szczególności przewidywanie wyniku przejawia się dużym zainteresowaniem ze względu na wspomnianą dostępność danych oraz wzrastającą liczbę zakładów bukmacherskich, z którymi można grać o prawdziwe pieniądze. To z kolei przyciąga nie tylko naukowców chcących tworzyć narzędzia i metody pozwalające na trafne określenia zwycięzcy meczu, ale też fanów tej dyscypliny, którzy stawiają swoje pieniądze z nadzieją na zysk dzięki podpowiadającej im wynik intuicji. 

Wymyślanie nowych rozwiązań i podejść leży więc w kwestii nie tylko twórców zakładów bukmacherskich, którzy, aby stale generować zyski, muszą ulepszać swoje metody predykcji, ale też między innymi działaczy klubu (w szczególności analityków) chcących zdawać sobie sprawę na temat tego, kto jest faworytem danego spotkania, aby w razie potrzeby móc odpowiednio dostosować taktykę na dany pojedynek, czy też hazardzistów korzystających z narzędzi predykcyjnych wspomagających ich grę.

Turnieje międzynarodowe w piłce nożnej są dobrą okazją dla naukowców chcących przetestować swoje rozwiązania. Ostatnie rozgrywki Mistrzostw Europy w 2016 obfitowały w prace prezentujące nowe podejścia w tej dziedzinie, co pokazuje jak duże zainteresowanie panuje w tym temacie~\cite{Euro2016-1} \cite{Euro2016-2}.

Przewidzenie zwycięzcy meczu jest bardzo trudnym zadaniem, gdyż wpływa na niego wiele czynników i większość z nich jest związana z czynnikiem ludzkim. Wyzwaniem jest więc wydobycie znaczących danych zarówno o meczu, jak i graczach biorących w nim udział, aby przewidzieć kto ostatecznie zwycięży w danym spotkaniu. Stale wymyślane są nowe podejścia oraz udoskonalane obecne. Praca ta podejmie próbę utworzenia skutecznego modelu, który będzie w stanie przewidzieć, która drużyna wyjdzie z danego pojedynku z tarczą. 

