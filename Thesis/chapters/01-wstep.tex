
\chapter{Wstęp: Sztuczna inteligencja w analizie danych dotyczących rozgrywek sportowych}

\noindent Praca podejmuje tematykę eksploracji danych dotyczących rozgrywek sportowych i wykorzystania algorytmów uczenia maszynowego do predykcji możliwych wyników tychże rozgrywek. Tematyka tak rozumianej analizy danych sportowych z jednej strony była i jest coraz intensywniej rozważana przez trenerów, a z drugiej strony stała się przedmiotem zainteresowania badaczy w ostatnich kilkunastu latach - co widać po rosnącej liczbie publikacji dotyczących zróżnicowanych dyscyplin sportowych (m.in piłki nożnej~\cite{Euro2016-1}, koszykówki~\cite{basketball}, hokeja na lodzie~\cite{ice-hockey}, kolarstwa~\cite{cyclists}, czy pływania~\cite{swimming}). 

Wśród badanych zastosowań powyższych metod można zwrócić uwagę na popularność wspomnianej już piłki nożnej \cite{Euro2016-1} \cite{Euro2016-2} \cite{Euro2016-3} \cite{soccer_players_skill} \cite{ml_soccer_analytics}. Także w niniejszej pracy zainteresowano się możliwością poszukania i pozyskania danych na temat rozgrywek meczów piłkarskich z ligi angielskiej -- najczęściej oglądanej w tej dyscyplinie sportu~\cite{ESPN} -- oraz zbadania przydatności różnych metod uczenia maszynowego w przewidywaniu rezultatu przyszłych meczów na podstawie historycznych danych z ostatnich sezonów. 

Zauważmy, ze popularność tej ligi i samej dyscypliny powoduje powstawanie ogromnej ilości danych możliwych do analiz, co czyni ją atrakcyjną dla badaczy. Szczególnym zainteresowaniem cieszy się przewidywanie wyniku ze względu na wspomnianą dostępność danych oraz wzrastającą liczbę zakładów bukmacherskich, które oferują grę o tzw. prawdziwe pieniądze. To z kolei przyciąga nie tylko naukowców chcących tworzyć narzędzia i metody pozwalające na trafne określenie zwycięzcy meczu, ale też fanów tej dyscypliny, którzy stawiają swoje oszczędności z nadzieją na wygraną dzięki  intuicji podpowiadającej im wynik. 

Poszukiwanie nowych rozwiązań i metod nie należy wyłącznie do działań  twórców zakładów bukmacherskich, którzy, aby stale generować zyski, muszą ulepszać swoje metody predykcji, ale też przede wszystkim dla analityków chcących przewidywać kto jest faworytem danego spotkania, po to aby w razie potrzeby trenerzy sportowi mogli odpowiednio dostosować taktykę na dany pojedynek czy dobrać zespół.

Ponadto studiując literaturę można zauważyć, że turnieje międzynarodowe w piłce nożnej stały się dobrym poligonem treningowym  dla naukowców chcących przetestować swoje rozwiązania. Ostatnie rozgrywki Mistrzostw Europy w 2016 roku wraz z udostępnieniem publicznie otwartych danych doprowadziły do powstania wielu artykułów naukowych  prezentujących nowe podejścia w tej dziedzinie, co pokazuje jak duże zainteresowanie budzi ten temat~\cite{Euro2016-1} \cite{Euro2016-2} \cite{Euro2016-3}.

Należy jednak zauważyć, że przewidzenie zwycięzcy meczu jest bardzo trudnym zadaniem, gdyż wpływa na niego wiele czynników. Większość z nich jest związana z czynnikiem ludzkim. 

Ponadto wyzwaniem jest zlokalizowanie właściwych repozytoriów danych, często bardzo zróżnicowanych (np. oprócz samych zapisów wyników kolejnych sezonów trzeba poszukiwać dodatkowych repozytoriów na temat samych graczy, oceny tzw. potencjału drużyny, a niektórzy autorzy sugerują wykorzystywanie różnych wskaźników stosowanych przez firmy bukmacherskie). Powyższe źródła danych są na ogół reprezentowane w różnych formatach, więc ich poprawna integracja wymaga wysiłku. Następnie należy zastosować odpowiednie metody wstępnego przetwarzania  w celu wydobycia najważniejszych elementów danych zarówno o meczu, jak i graczach biorących w nim udział, po to aby przewidzieć kto ostatecznie zwycięży w danym spotkaniu. Same algorytmy uczenia maszynowego były najczęściej rozwijane dla innych zastosowań, a wyniki zamieszczone w literaturze nie są często o wysokiej trafności. Ponadto zauważamy, że ciągle tworzone są nowe rozwiązania oraz udoskonalane obecne algorytmy. 

Kierując się powyższymi motywacjami w niniejszej rozprawie podjęto próbę utworzenia całościowego systemu, który będzie nie tylko integrował różnorodne dane o rozgrywkach piłkarskich ligi angielskiej, a przede wszystkim stosował wybrane algorytmy uczenia maszynowego do przewidywania, która drużyna zostanie zwycięzcą kolejnego pojedynku. 

